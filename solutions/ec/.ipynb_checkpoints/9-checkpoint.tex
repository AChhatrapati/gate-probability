Given, a fair coin is tossed is tossed two times.
Let's define a Markov chain $\{X_{n},n=0,1,2,\dots\}$, where $X_{n}\in S=\{1,2,3\}$, such that
\begin{table}[h!]
\centering
\caption{States and their notations}
\label{ec9:table:1}
\begin{tabular}{|c|c|}
    \hline
    Notation & State \\
    \hline
    $S=1$ & getting \cbrak{TT}\\[1ex]
    \hline
    $S=2$ & getting output Y\\[1ex]
    \hline
    $S=3$ & getting output N\\[1ex]
    \hline
\end{tabular}
\end{table}\\
The state transition matrix for the Markov chain is
\begin{align}
\label{ec9:eq:1}
P=\begin{blockarray}{cccccc}
&1 & 2 & 3 \\
\begin{block}{c[ccccc]}
  1 & 0.25 & 0.25 & 0.5  \\
  2 & 0 & 1 & 0 \\
  3 & 0 & 0 & 1  \\
\end{block}
\end{blockarray}
\end{align}
Clearly, the state $1$ are transient, while $2,3$ are absorbing. The standard form of a state transition matrix is
\begin{align}
\label{ec9:eq:2}
P=\begin{blockarray}{ccc}
&A & N \\
\begin{block}{c[cc]}
  A & I & O  \\
  N & R & Q \\
\end{block}
\end{blockarray}
\end{align}
where,
\begin{table}[h!]
\centering
\caption{Notations and their meanings}
\label{ec9:table:2}
\begin{tabular}{|c|c|}
    \hline
    Notation & Meaning \\
    \hline
    $A$ & All absorbing states\\[1ex]
    \hline
    $N$ & All non-absorbing states\\[1ex]
    \hline
    $I$ & Identity matrix\\[1ex]
    \hline
    $O$ & Zero matrix\\[1ex]
    \hline
    $R,Q$ & Other submatices\\[1ex]
    \hline
\end{tabular}
\end{table}
Converting \eqref{ec9:eq:1} to standard form, we get
\begin{align}
P=\begin{blockarray}{cccccc}
&2 & 3 & 1  \\
\begin{block}{c[ccccc]}
  2 & 1 & 0 & 0  \\
  3 & 0 & 1 & 0 \\
  1 & 0.25 & 0.5 & 0.25  \\
\end{block}
\end{blockarray}
\end{align}
From \eqref{ec9:eq:2},
\begin{align}
\tag{104.5}
\label{ec9:eq:r,q}
    R=\begin{bmatrix}
    0.25 & 0.5\\
    \end{bmatrix},
    Q=\begin{bmatrix}
    0.25\\
    \end{bmatrix}
\end{align}
The limiting matrix for absorbing Markov chain is
\begin{align}
\bar P=\begin{bmatrix}
    I & O\\
    FR & O\\
    \end{bmatrix}
\end{align}
where,
\begin{align}
F=(I-Q)^{-1}
\end{align}
is called the fundamental matrix of $P$. \\
On solving, we get
\begin{align}
\bar P=\begin{blockarray}{cccccc}
&2 & 3 & 1 \\
\begin{block}{c[ccccc]}
    2 & 1 & 0 & 0\\ 
    3 & 0 & 1 & 0\\ 
    1 & 0.33 & 0.17 & 0\\    
\end{block}
\end{blockarray}
\end{align}
A element $\bar p_{ij}$ of $\bar P$ denotes the absorption probability in state $j$, starting from state $i$. 
\\ Then, the absorption probability in state 2 $\brak{\text{i.e getting output Y}}$ starting from state 1 is $\bar p_{12}$.
\begin{align}
\therefore \bar p_{12}=0.33 \brak{\text{correct upto 2 decimal places}}
\end{align}
\begin{figure}[h]
\caption*{\textbf{Markov chain diagram}}
\centering
\begin{tikzpicture}
       
             % Setup the style for the states
        \tikzset{node style/.style={state, 
                                    minimum width=2cm,
                                    line width=1mm,
                                    fill=gray!20!white}}
        % Draw the states
        \node[node style] at (3, 0)      (bull)     {$X_{1}$};
        \node[node style] at (0, -4)      (bear)     {$X_{2}$};
        \node[node style] at (6, -4) (stagnant) {$X_{3}$};
        % Connect the states with arrows
        \draw[every loop,
              auto=right,
              line width=0.7mm,
              >=latex,
              draw=orange,
              fill=orange]
           (bull)     edge[bend left=20]            node {$\frac{1}{2}$} (stagnant)
            (bull)     edge[bend right=20] node {$\frac{1}{4}$} (bear)
            
            
            (bull) edge[loop above]             node  {$\frac{1}{4}$} (bull)
            (bear) edge[loop below]             node  {1} (bear)
            (stagnant) edge[loop below]             node  {1} (stagnant);
    \end{tikzpicture}
\end{figure}