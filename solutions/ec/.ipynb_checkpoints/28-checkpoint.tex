The values that the random variable X can take along with its probabilities are given by
\begin{table}[h]
\centering
\begin{tabular}{|l|l|l|l|}
\hline
X             & 0             & 1             & 2             \\ \hline
$\Pr\brak{X}$ & $\frac{1}{2}$ & $\frac{1}{4}$ & $\frac{1}{4}$ \\ \hline
\end{tabular}
\end{table}
\newline
The values that the random variable Y can take along with its probabilities are given by
\begin{table}[h]
\centering
\begin{tabular}{|l|l|l|l|}
\hline
Y             & 0             & 1             & 2             \\ \hline
$\Pr\brak{Y}$ & $\frac{1}{2}$ & $\frac{1}{4}$ & $\frac{1}{4}$ \\ \hline
\end{tabular}
\end{table}
\begin{align}
\Pr\brak{X-Y=0}=\frac{1}{2}\times\frac{1}{2}+\frac{1}{4}\times\frac{1}{4}+\frac{1}{4}\times\frac{1}{4}=\frac{6}{16}\\
\Pr\brak{(X+Y=2),(X-Y=0)}=\frac{1}{4}\times\frac{1}{4}=\frac{1}{16}
\end{align}
\begin{align}
\Pr\brak{X+Y=2\:|\:X-Y=0}\notag\\
=&\frac{\Pr\brak{(X+Y=2),(X-Y=0)}}{\Pr\brak{X-Y=0}}\notag\\
=&\frac{\frac{1}{16}}{\frac{6}{16}}=\frac{1}{6}
\end{align}