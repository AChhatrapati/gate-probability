Given that the coin is tossed until a head appears on an odd toss.\\
\begin{align}
    p = \frac{1}{2}, q = \frac{1}{2}
\end{align}
 Let's define a Markov chain $\{X_{n},n=0,1,2,\dots\}$, where $X_{n}\in S=\{1,2,3,4\}$, such that:
\begin{table}[h!]
\centering
\caption{States and their notations}
\label{ec25:table:1}
\begin{tabular}{|c|c|}
    \hline
    Notation & State \\
    \hline
    $S=1$ & Odd try\\[1ex]
    \hline
    $S=2$ & Even try\\[1ex]
    \hline
    $S=3$ & Loss\\[1ex]
    \hline
    $S=4$ & Success\\[1ex]
    \hline
\end{tabular}
\end{table}
\\The state transition matrix for the Markov chain is:
\begin{align}
\tag{2.0.1}
\label{ec25:eq:2.0.1}
    P=\begin{blockarray}{ccccc}
    &1 &2 &3 &4\\
    \begin{block}{c[cccc]}
1 & 0 & 0.5 & 0 & 0.5\\
2 & 0.5 & 0 & 0.5 & 0\\
3 & 0 & 0 & 1 & 0\\
4 & 0 & 0 & 0 & 1\\
\end{block}
\end{blockarray}
\end{align}
The transient states are 1,2 and the absorbing states are 3 and 4. The standard form of the matrix is;
\begin{align}
\tag{2.0.2}
\label{ec25:eq:2.0.2}
    P=\begin{blockarray}{ccc}
&A & N \\
\begin{block}{c[cc]}
  A & I & O  \\
  N & R & Q \\
\end{block}
\end{blockarray}
\end{align}
where,
\begin{table}[h!]
\centering
\caption{Notations and their meanings}
\label{ec25:table:2}
\begin{tabular}{|c|c|}
    \hline
    Notation & Meaning \\
    \hline
    $A$ & Absorbing states\\[1ex]
    \hline
    $N$ & Non-absorbing states\\[1ex]
    \hline
    $I$ & Identity matrix\\[1ex]
    \hline
    $O$ & Zero matrix\\[1ex]
    \hline
    $R,Q$ & Other sub-matrices\\[1ex]
    \hline
\end{tabular}
\end{table}
\\Now, we convert the transition matrix to this standard form.
\begin{align}
\tag{2.0.3}
\label{ec25:eq:2.0.3}
    P = \begin{blockarray}{ccccc}
    &3 &4 &1 &2\\
    \begin{block}{c[cccc]}
    3 & 1 & 0 & 0 & 0 \\
    4 & 0 & 1 & 0 & 0 \\
    1 & 0 & 0.5 & 0 & 0.5 \\
    2 & 0.5 & 0 & 0.5 & 0 \\
    \end{block}
    \end{blockarray}
\end{align}
From \eqref{ec25:eq:2.0.3}, 
\begin{align}
    \tag{2.0.4}
    \label{ec25:eq:2.0.4}
    R = \begin{bmatrix}
    0 & 0.5\\
    0.5 & 0\\
    \end{bmatrix}
    , Q = \begin{bmatrix}
    0 & 0.5\\
    0.5 & 0\\
    \end{bmatrix}
\end{align}
\\The limiting matrix for absorbing Markov chain is,
\begin{align}
\tag{2.0.5}
\label{ec25:eq:2.0.5}
    \bar P=\begin{bmatrix}
    I & O\\
    FR & O\\
    \end{bmatrix}
\end{align}
where,
\begin{align}
\tag{2.0.6}
\label{ec25:eq:2.0.6}
    F=(I-Q)^{-1}
\end{align}
is called the fundamental matrix of $P$. \\
On solving we get,
\begin{align}
    \tag{2.0.7}
    \label{ec25:eq:2.0.7}
    \bar P = \begin{blockarray}{ccccc}
    &3 &4 &1 &2\\
    \begin{block}{c[cccc]}
    3 & 1 & 0 & 0 & 0\\
    4 & 0 & 1 & 0 & 0\\
    1 & 0.333 & 0.667 & 0 & 0\\
    2 & 0.667 & 0.333 & 0 & 0\\
    \end{block}
    \end{blockarray}
\end{align}
An element $\bar p_{ij}$ of $\bar P$ denotes the absorption probability to the state $j$, starting from the state $i$.
\\Let $\pr{A}$ be the probability that the first head is obtained on an odd toss. Then,
\begin{align}
    \pr{A} &= p_{14}\\
    &= 0.667\\
    \therefore \pr{A} &= \frac{2}{3}
\end{align}
\begin{figure}[h]
\caption*{\textbf{Markov chain diagram}}
\centering
\begin{tikzpicture}
    % Setup the style for the states
        \tikzset{node style/.style={state, 
                                    minimum width=1.2cm,
                                    line width=0.85mm,
                                    fill=gray!20!white}}
        % Draw the states
        \node[node style] at (3, -3)      (bull)     {1};
        \node[node style] at (0, -6)      (bear)     {2};
        \node[node style] at (6, -6) (stagnant) {4};
        \node[node style] at (3, -9)
        (loss) {3};
        % Connect the states with arrows
        \draw[every loop,
              auto=right,
              line width=0.7mm,
              >=latex,
              draw=red,
              fill=red]
            (stagnant)     edge[loop right]            node {1} (stagnant)
            (bull)     edge[bend right=20] node {$\dfrac{1}{2}$} (bear)
            (bear)     edge[bend right=20] node {$\dfrac{1}{2}$} (bull)
            (bull)     edge[bend left=20] node {$\dfrac{1}{2}$} (stagnant)
            (bear) edge[bend right=20]
            node{$\dfrac{1}{2}$} (loss)
            (loss) edge[loop right] node{1} (loss);
\end{tikzpicture}
\end{figure}