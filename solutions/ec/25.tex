We can see that if the first toss is guaranteed to be a head, then the problem is reduced to finding the probability of getting one head in 2 coin tosses, since all the 3 trials are independent.
Let $K=\{0, 1, 2\}$ be the random variable denoting the number of heads obtained in 2 tosses of a fair coin. The event can be represented by a binomial distribution b(n,p).
In binomial distribution b(n,p), 
\begin{align}
Pr\brak{K=i}= \binom{n}{i}p^i \cdot (1-p)^{n-i}.
\end{align}
Here $n=2, p=0.5$.
 We can see that the probability of $K=1$ is 
 \begin{align}
 Pr\brak{K=1}&=\binom{2}{1} \cdot 0.5^2\\
 &=\frac{1}{2}
 \end{align}
 
From $\brak{2.0.1}$ and $\brak{2.0.2}$, we see that probability of getting exactly 2 heads in 3 tosses, if the first toss is a head, is 0.5.