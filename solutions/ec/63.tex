We know that
\begin{align}
var(X)&=E[(X-E[X])^2] \label{63:eq:1} \\
var(X)&=E[X^2]-(E[X])^2 \label{63:1}
\end{align}
For uniform distribution in the interval $[a,b]$
\begin{align}
var(X) &= \frac{{(b-a)}^2}{12} \label{63:eq:2}
\end{align}
For uniform distribution,$(b-a)^2 \geq 0$\\
By definition of variance,it is average value of ${(X-E[X])}^2$.\\
Since ${(X-E[X])}^2 \geq 0$ ,average $E[(X-E[X])^2] \geq 0$.
\begin{align}
\therefore var(X) & \geq 0 \label{63:2} \\
\therefore E[X^2]-(E[X])^2 & \geq 0 \label{63:3}
\end{align}
\begin{enumerate}
\item  $E[X]=0$ and $E[X^2]=1$
\begin{align}
E[X^2]-(E[X])^2 &=1 - 0\\
&=1\\
\therefore E[X^2]-(E[X])^2 &\geq 0
\end{align}
$\therefore$ $E[X]=0$ and $E[X^2]=1$ can be attained \\
\item  $E[X]=\frac{1}{2}$ and $E[X^2] =\frac{1}{3}$
\begin{align}
E[X^2]-(E[X])^2 &=\frac{1}{3} - \frac{1}{4}\\
&=\frac{1}{12}\\
\therefore E[X^2]-(E[X])^2 &\geq 0
\end{align}
$\therefore$ $E[X]=\frac{1}{2}$ and $E[X^2]=\frac{1}{3}$ can be attained \\
\item  $E[X]=2$ and $E[X^2]=3$
\begin{align}
E[X^2]-(E[X])^2 &=3 - 4\\
&=-1\\
\therefore E[X^2]-(E[X])^2 &\leq 0
\end{align}
$\therefore$ $E[X]=2$ and $E[X^2]=3$ cannot be attained \\
\item  $E[X]=2$ and $E[X^2]=5$
\begin{align}
E[X^2]-(E[X])^2 &=5 - 4\\
&=1\\
\therefore E[X^2]-(E[X])^2 &\geq 0
\end{align}
$\therefore$ $E[X]=2$ and $E[X^2]=5$ can be attained\\ \\
$\therefore$ $E[X]=2$ and $E[X^2]=3$ cannot be attained\\
\end{enumerate}