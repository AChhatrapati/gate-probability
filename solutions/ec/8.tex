Let X be random variable.\\
X $\in$ \{1,2,3,4,5,6\}\\
$p_{X}(n)\rightarrow$Probability of showing up n.\\
As $p_{X}(n)$ is proportional to n. We have,
\begin{align}
    p_{X}(n) = 
    \begin{cases}
        kn & 1 \leq n \leq 6 \\
        0  & otherwise
    \end{cases}
    \tag{8.1}
\end{align}
Where k is some real constant.

\begin{table}[ht]
\centering 
\caption{}
\begin{tabular}{|c|c|c|c|c|c|c|}
\hline
 n          & 1 & 2 & 3 & 4 & 5 & 6 \\
\hline
 $p_{X}(n)$ & k  & 2k & 3k & 4k & 5k & 6k\\
\hline
\end{tabular}
\label{8:table}
\end{table}

We know that,
\begin{align}
    \tag{8.2}
    \sum_{n=1}^6 p_{X}(n)=1
    \label{8:eq:1}
\end{align}
By substituting the values in \ref{8:eq:1}, we have
\begin{align}
    \tag{8.3}
    k+2k+3k+4k+5k+6k=1
\end{align}
\begin{align}
     \tag{8.4}
   \implies  k=\frac{1}{21}
   \label{8:eq:2}
\end{align}
Probability of the face with three dots showing up 
\begin{align}
    \tag{8.5}
    \implies p_{X}(3)=3k
\end{align}
Substituting the value of k from \ref{8:eq:2}
\begin{align}
    \tag{8.6}
    \implies p_{X}(3)=\frac{1}{7}
\end{align}