For a continuous joint probability distribution  $\e{X}$ \\
and $\e{Y}$ are obtained using the following equations  \\
\eqref{ec78:a} and \eqref{ec78:b}
\begin{align}
\e{X} &= \Int_{-\infty}^{+\infty}\Int_{-\infty}^{+\infty} x \cdot \fn{x,y}\,dx\,dy \label{ec78:a} \\ 
\e{Y} &= \Int_{-\infty}^{+\infty}\Int_{-\infty}^{+\infty} y \cdot \fn{x,y}\,dx\,dy \label{ec78:b}
\end{align}
Using equation \eqref{ec78:a} \e{X} is calculated as 
\newpage
\begin{align*}
\e{X} &= \Int_{0}^{1}\Int_{0}^{1} x\,\dfrac{6}{5}\brak{x+y^2}\,dx\,dy \;+ 0 \\ 
      &= \Int_{0}^{1}\dfrac{6}{5}\brak{\Int_{0}^{1}x^2\,dx}+\dfrac{6}{5}\,y^2\,\brak{\Int_{0}^{1}x\,dx}\;dy   \\
      &= \Int_{0}^{1}\dfrac{6}{5}\brak{\dfrac{1}{3}}+\dfrac{6}{5}\,y^2\,\brak{\dfrac{1}{2}}\;dy \\
      &= \dfrac{2}{5}\Int_{0}^{1}\,dy + \dfrac{3}{5}\Int_{0}^{1}y^2\,dy  \\
      &= \dfrac{2}{5} + \dfrac{3}{5}\brak{\dfrac{1}{3}} \\
\e{X} &=  \dfrac{3}{5}  
\end{align*}
Using equation \eqref{ec78:b} \e{Y} is calculated as
\begin{align*}
\e{Y} &= \Int_{0}^{1}\Int_{0}^{1} y\,\dfrac{6}{5}\brak{x+y^2}\,dx\,dy \;+ 0 \\ 
      &= \Int_{0}^{1}\dfrac{6}{5}\,x\brak{\Int_{0}^{1}y\,dy} + \dfrac{6}{5}\brak{\Int_{0}^{1}y^{3}\,dy}\;dx \; \\ 
      &= \Int_{0}^{1}\dfrac{6}{5}\,x\brak{\dfrac{1}{2}} + \dfrac{6}{5}\brak{\dfrac{1}{4}}\;dx \; \\ 
      &= \dfrac{3}{5}\Int_{0}^{1}x\,dx\;+\;\dfrac{3}{10}\Int_{0}^{1}\,dx  \\
      &= \dfrac{3}{5} \brak{\dfrac{1}{2}} + \dfrac{3}{10} \\
\e{Y} &= \dfrac{3}{5}      
\end{align*}
 $$\therefore \e{X} = \dfrac{3}{5}\;\text{and}\;\e{Y} = \dfrac{3}{5}$$ 
 Hence the answer is \textbf{option b}