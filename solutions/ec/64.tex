We know, when one dice is rolled probability i.e \pr{X_{1}=r} for all r in \{1,2,3,4,5,6\} is equal to  p
\begin{align}
    p&=\frac{1}{6}
\end{align}
Let $Y_{i}$ denote the value obtained on ith dice when n dices are rolled  
, therefore 
\begin{align}
  X_{n}&=\sum_{i=1}^n Y_{i}
  \label{ec64:eq:eq1}
\end{align}
Now i will calculate expectation value of value obtained when one dice is rolled
using below formula;
\begin{align}
 E(Y_{i})&= E( X_{1}) =\sum_{r=1}^6 (r\times p)
 \label{ec64:eq:eq2}
\\
&=\frac{1}{6} \times \sum_{r=1}^6 r
\\
&=\frac{7}{2}.
\label{ec64:eq:eq3}
\end{align}
\begin{enumerate}
\item Since the Expectation value of a sum of independent events is the sum of their expectation. So,
\begin{align}
    E(X_{n})&=\sum_{i=1}^n E(Y_{i})
    \\
    & = \sum_{i=1}^n \frac{7}{2} =\frac{7}{2} n 
\label{ec64:eq:eq4}
\end{align}
\item By Using the following formula ,we can calculate variance of  $X_{1}$ ,
\begin{align}
    V(X_{1})&=(E(X_{1})^{2}) - (E(X_{1}))^{2}
    \label{ec64:eq:eq5}
    \\
    \sum_{i=1}^k r^2&=\frac{k\times (k+1 )\times (2(k)+1)}{6}
    \label{ec64:eq:eq6}
\end{align}
Now calculating E($X_{1}^{2}$),by using \eqref{ec64:eq:eq6}
\begin{align}
    E(X_{1}^{2})&=\sum_{r=1}^6(r^{2}\times p)
    \\
    &=\frac{1}{6}\times\sum_{r=1}^6r^{2}
    \\
    &=\frac{91}{6}
    \label{ec64:eq:eq7}
\end{align}
By using \eqref{ec64:eq:eq3},\eqref{ec64:eq:eq5} and \eqref{ec64:eq:eq7}
\begin{align}
    V(X_{1})&=V(Y_{i})
    \\
    &=\frac{35}{12}
    \label{ec64:eq:eq8}
\end{align}
Variance of sum can be calculated by using following formula,
\begin{align}
    V(X_{n})&=V(\sum_{i=0}^n Y_{i})
    \\
    &= \sum_{i=1}^n V(Y_{i}) + \sum_{1\leq i\not=j \leq n}\text{Cov}(Y_{i},Y_{j})
\end{align}
Since Co-variance of independent random variables is zero.So,
\begin{align}
    V(X_{n})&=\sum_{i=1}^n V(Y_{i}) + 0
    \\
    &=\frac{35}{12}n
\end{align}
\end{enumerate}
    Hence option(B) is correct.