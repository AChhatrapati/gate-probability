
Let M,P,C be the events representing student passes in Mathematics,Physics,Chemistry respectively.
\begin{align}
    \pr{M}&=m\\
    \pr{P}&=p\\
    \pr{C}&=c
\end{align}
The given information can be represented as
\begin{align}
    \pr{M+P+C}=75\%=\dfrac{3}{4}\label{atleast_1}\\
    \pr{MP+PC+CA}=50\%=\dfrac{1}{2}\label{atleast_2}\\
    \pr{MP+PC+CA-3MPC}=40\%=\dfrac{2}{5} \label{only_2}
\end{align}
\eqref{atleast_2} and \eqref{only_2} can also be written as
\begin{align}
    \pr{MP}+\pr{PC}+\pr{CM}&\nonumber\\-2\pr{MPC}&=\dfrac{1}{2}\\
    \pr{MP}+\pr{PC}+\pr{CM}&\nonumber\\-3\pr{MPC}&=\dfrac{2}{5}
\end{align}
Subtracting and solving the above two equations we get,
\begin{align}
    \pr{MPC}&=\dfrac{1}{10}\\
    \pr{MP}+\pr{PC}+\pr{CM}&=\dfrac{7}{10}
\end{align}
Using inclusion-exclusion principle, We can express \eqref{atleast_1} as
\begin{align}
\pr{M}+\pr{P}+\pr{C}&\nonumber\\
-[\pr{MP}+\pr{PC}+\pr{CM}]&\nonumber\\
      +\pr{MPC}&=\dfrac{3}{4}\\
    p+m+c-\dfrac{7}{10}+\dfrac{1}{10}&=\dfrac{3}{4}\\
    p+m+c&=\dfrac{27}{10}
\end{align}
There is no constant answer for the product of p,m,c which is shown in simulation.\\\\
\centering $\therefore$ Only relation I is true.