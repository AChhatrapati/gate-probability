For two random events A and B that are independent, we know that, 
\begin{align}
Pr(AB) = Pr(A)Pr(B)
\end{align}
and for two mutually exclusive events C and D, 
\begin{align}
    Pr(CD) = 0
\end{align}

\begin{enumerate}[label = (\alph*)]
    \item Independence of P and Q implies that the occurrence of one is unaffected by the other. 
    \begin{align}
       \Rightarrow Pr(PQ) = Pr(P)Pr(Q)
    \end{align}
    The given option will be true only when either Pr(P) or Pr(Q) will be zero, therefore, (a) is incorrect.\\
    \item From set theory,
    \begin{align}
    A\cup B &= A + B - A\cap B
    \end{align}
    \begin{multline}
    \Rightarrow Pr(P+Q) = Pr(P) \\+ Pr(Q)- Pr(PQ)
    \label{ee2005:eq:5}
    \end{multline}
    \begin{align}
    \Rightarrow Pr(P+Q) &\leq Pr(P) + Pr(Q)
    \end{align}
    thus, (b) is incorrect.\\
    \item Two events can be both mutually exclusive and independent only when one of them have a zero probability. Since it isn't necessary that $Pr(P)=0$ or $Pr(Q)=0$, (c) is incorrect.\\
    \item The set $P$ will have either have the same or more elements than the set $P\cap Q$
    \begin{equation}
        Pr(PQ) \leq Pr(P)
    \end{equation}
    (d) is correct.\\
\end{enumerate}
Thus, the only correct option is (d).