
Let the random variable X represent which player gets six first. That is $X=0 $ when A gets a six first and $X=1 $ when B gets six first. \\
Let another random variable Y represent getting a six on the dice. $Y=1 $ for six and $Y=0 $ for any other number. \\
Let N be the number of turns until we get a six.
\begin{align}
    \Pr\brak{Y=0} = \frac{5}{6} \\
    \Pr\brak{Y=1} = \frac{1}{6}
\end{align}
The event success is when B gets a six for first time and failure is when neither A nor B gets six. Let p denote probability of success 
\begin{align}
    p &= \Pr\brak{Y=1}  \\
    \Pr\brak{Y=0} &= 1- p \\
    p &= \frac{1}{6}
\end{align}
To get $X=1 $ in $N$ turns we have to get $N-1$ failures for B and $ N$ failures for A and finally one success for B. 
Therefore the geometric distribution is,
\begin{align}
    f(N) &= \brak{1-p}^{n-1} \times p \times  \brak{1-p}^{n} \\
    &  = \brak{1-p}^{2n-1} \times p \\
    & = \brak{ \frac{5}{6}}^{2n-1} \times \frac{1}{6}
\end{align}
The result has been summarized in table \ref{tab:table geometric distribution}. \\
\begin{table}[hbt!]
\centering
\begin{tabular}{|c|c|}
\hline
\textbf{No. of turns} & \textbf{Probability} \\ \hline
1                     & $5^1/6^2  $            \\ \hline
2                     & $5^3/6^4 $             \\ \hline
$\vdots  $              & $\vdots $              \\
n                     & $5^{2n-1}/6^{2n} $         \\ \hline
$\vdots $               & $\vdots $              \\ \hline
\end{tabular}
\caption{Summary of turns}
\label{tab:table geometric distribution}
\end{table}
Thus the total probability is sum of these individual probabilities i.e.
\begin{align}
    \Pr\brak{X=1} &= \sum_{N=1}^{\infty} f(N) \\
    &= \frac{5}{6^2} + \frac{5^3}{6^4} + \hdots + \frac{5^{2n-1}}{6^{2n}} +\hdots  \\
    &= \frac{5}{6^2} \times \brak{1 + \frac{5^2}{6^2} + \frac{5^4}{6^4} + \hdots } 
\end{align}
By Using sum of infinite GP we have, 
\begin{align}
     \Pr\brak{X=1} &= \frac{5}{6^2} \times \brak{\frac{1}{1 - \frac{25}{36}}}  \\
       &= \frac{5}{36} \times \frac{36}{11} \\
       &= \frac{5}{11} = 0.45
\end{align}