This problem can be solved using Queuing theory.But first we have to understand queuing theory.
\begin{itemize}
\item In queuing theory we try to determine what happens when people join in queue.
\item\textbf{Parameters for measuring Queuing performance}
\begin{enumerate}
    \item $\lambda$ = Average arrival time
    \item $\mu$ = Average service time
    \item $\rho$ = Utilization factor
    \item $L_q$ = Average number in the queue
    \item $L$ = Average number in the system
    \item $W_q$ = Average waiting time
    \item $W$ = Average time in the system
    \item $P_n$ = Steady state probability of exactly n customers in the system
\end{enumerate}
\item Typically most of the times arrivals follow poisson distribution and services follow exponential distributions.
\item The given question only has one queue so we can conclude that it is a single server model and there is no limit for number of cars in the queue so we can say that it is "\textbf{M/M/1:/$\infty$/$\infty$/FIFO}" by kendall's notation (or) usually "\textbf{M/M/1}" 
\item Here '\textbf{M}' indicates the memory less property of the model first \textbf{M} is for arrival and second one for service and 1 is the number of servers in the model and '$\infty$' indicates the limit of the queue and second '$\infty$' represent population and '\textbf{FIFO}' represents First-In First-Out service.
\item \textbf{NOTE:} In cases where there is no limit in the queue we only take the cases where $\frac{\lambda}{\mu}<1$. Otherwise there could be customers who will not get their service. \newline The memory less property allows us to assume that one event can take place in a small interval of time. The event could be either a arrival or a service.
\item\textbf{Deriving formulas :}
 For the time interval($t,t+h$), where $h \to 0$
\begin{align}
    \Pr{(\text{1 arrival})}&=\lambda h\\
    \Pr{(\text{1 service})}&=\mu h\\
    \Pr{(\text{no arrival})}&=1-\lambda h\\
    \Pr{(\text{no service})}&=1-\mu h
\end{align}
\begin{multline}
    P_{n}(t+h)=P_{n-1}(t)\times\Pr{(\text{1 arrival})}\times\Pr{(\text{no service})}\\
    +P_{n+1}(t)\times\Pr{(\text{no arrival})}\times\Pr{(\text{1 service})}\\
    +P_n(t)\times\Pr{(\text{no arrival})}\times\Pr{(\text{no service})}\label{me2011-19:eq:eq1}
\end{multline}
\begin{multline}
    \implies P_{n}(t+h)=P_{n-1}(t)(\lambda h)(1-\mu h)\\
    +P_{n+1}(t)(\mu h)(1- \lambda h)\\
    +P_n(t)(1-\lambda h)(1-\mu h)
\end{multline}
Now, Neglecting higher order terms of $h$.
\begin{multline}
    \implies P_{n}(t+h)=P_{n-1}\lambda h+P_{n+1}\mu h\\
    +P_n(t)(1-\lambda h-\mu h)
\end{multline}
\begin{multline}
    \implies \frac{P_n(t+h)-P_n(t)}{h}=P_{n-1}(t)\lambda+P_{n+1}(t)\mu\\
    -P_n(t)(\lambda+\mu)
\end{multline}
At steady state, $P_n(t+h)=P_n(t)$
\begin{align}
    \implies\lambda P_{n-1}+\mu P_{n+1}&=(\lambda+\mu)P_n\label{me2011-19:eq:res1}
\end{align}
Now, calculating $P_0(t+h)$ using \eqref{me2011-19:eq:eq1}
\begin{multline}
    P_0(t+h)=P_1(t)(1-\lambda h)(\mu h)\\
    +P_0(t)(1-\lambda h)
\end{multline}
Again, Neglecting higher order terms of $h$
\begin{multline}
    \implies P_0(t+h)=P_1(t)(\mu h)\\
    +P_0(t)(1-\lambda h)
\end{multline}
\begin{align}
    \implies\frac{P_0(t+h)-P_0(t)}{h}&=P_1(\mu)-P_0(\lambda)
\end{align}
At steady state, $P_0(t+h)=P_0(t)$
\begin{align}
    \implies \mu P_1&=\lambda P_0\label{me2011-19:eq:eq2}\\
    \implies P_1&=\brak{\frac{\lambda}{\mu}}P_0\label{me2011-19:eq:res2}
\end{align}
Using \eqref{me2011-19:eq:res1} by substituting $n=1$
\begin{align}
    \lambda P_0+\mu P_2&=(\lambda+\mu)P_1\\
    \implies \lambda P_0+\mu P_2&=\lambda P_1+\mu P_1
\end{align}
from \eqref{me2011-19:eq:eq2} and \eqref{me2011-19:eq:res2}
\begin{align}
    \implies\lambda P_0+\mu P_2&=\lambda P_1+\lambda P_0\\
    \implies P_2&=\brak{\frac{\lambda}{\mu}}P_1\\
    \implies P_2&=\brak{\frac{\lambda}{\mu}}^2P_0\label{me2011-19:eq:eq3}
\end{align}
We assume $\frac{\lambda}{\mu}=\rho$ and generalize $P_n$ by \eqref{me2011-19:eq:res2} and \eqref{me2011-19:eq:eq3}
\begin{align}
  P_n&=\brak{\frac{\lambda}{\mu}}^nP_0\\
  \implies P_n&=\rho^nP_0\label{me2011-19:eq:res3}
\end{align}
We know that sum of all probabilities equal to 1
\begin{align}
    \sum_{i=1}^{\infty}P_i&=1\\
    \implies P_0+P_1+P_2+....&=1
\end{align}
Using \eqref{me2011-19:eq:res3}
\begin{align}
    \implies P_0+\rho P_0+\rho^2P_0+....&=1\\
    \implies P_0\brak{1+\rho+\rho^2+...}&=1\\
    \implies P_0\brak{\frac{1}{1-\rho}}&=1\\
    \implies P_0&=1-\rho\label{me2011-19:eq:res4}\\
    \therefore P_n=\rho^n(1-\rho)
\end{align}
The number of people in the system ($L_s$) is the expected value
\begin{align}
    L_s&=\sum_{i=0}^{\infty}iP_i\\
    \implies L_s&=\sum_{i=0}^{\infty}i\rho^iP_0\\
    \implies L_s&=\rho P_0\sum_{i=0}^{\infty}i\rho^{i-1}\\
    \implies L_s&=\rho P_0\sum_{i=0}^{\infty}\frac{d}{d\rho}\brak{\rho^i}\\
    \implies L_s&=\rho P_0\frac{d}{d\rho}\sum_{i=0}^{\infty}\rho^i\\
    \implies L_s&=\rho P_0\frac{d}{d\rho}\brak{\frac{1}{1-\rho}}\\
    \implies L_s&=\rho P_0\frac{1}{(1-\rho)^2}
\end{align}
By using \eqref{me2011-19:eq:res4}
\begin{align}
    \implies L_s&=\rho(1-\rho)\frac{1}{(1-\rho)^2}\\
    \implies L_s&=\frac{\rho}{1-\rho}
\end{align}
We can also say that the number of people beign served is $\rho$
\begin{align}
    \therefore L_s&=L_q+\text{people beign served}\\
    \implies L_s&=L_q+\rho\\
    \implies L_q&=L_s-\rho\\
    \implies L_q&=\frac{\rho}{1-\rho}-\rho\\
    \implies L_q&=\frac{\rho^2}{1-\rho}
\end{align}
\newpage
The relation between $L_s$ and $W_s$ and $L_q$ and $W_q$ are the Little's equation and they are related as
\begin{align}
    L_s&=\lambda W_s\\
    L_q&=\lambda W_q
\end{align}
\end{itemize}
\section{Solution}
From the question given,
\begin{align}
    \lambda &=5\text{hr}^{-1}\\
    \mu&=\frac{1}{10}\text{min}^{-1} =6\text{hr}^{-1}
    \end{align}
Therefore,
\begin{align}
    \text{Utilization rate}(\rho)&=\frac{\lambda}{\mu}=\frac{5}{6}
    \end{align}
Average number (or) length in queue be $L_q$
    \begin{align}
    L_q&=\frac{\rho^2}{1-\rho}\\
    &=\frac{\brak{\frac{5}{6}}^2}{1-\frac{5}{6}}\\
    &=\frac{25}{6}
    \end{align}
Let the Average waiting time in queue be $W_q$
    \begin{align}
    W_q&=\frac{L_q}{\lambda}\\
    &=\frac{\frac{25}{6}}{5}\\
    &=\frac{5}{6}\text{hr}=50\text{min}
\end{align}
The average waiting time in the queue is 50 min.
\newpage
\begin{table}[h!]
\centering
\resizebox{\columnwidth}{!}
{
    \begin{tabular}{|c|c|}
    \hline
    Parameter & Value \\
    \hline
    $\lambda$ & $5$hr$^{-1}$\\
    \hline
    $\mu$ & $6$hr$^{-1}$\\
    \hline
    $\text{Utilization rate}\brak{\rho}=\frac{\lambda}{\mu}$ & $\frac{5}{6}$\\[1ex]
    \hline
    $\text{Length in queue}\brak{L_q}=\frac{\rho^2}{1-\rho}$ & $\frac{25}{6}$\\[1ex]
    \hline
    $\text{Waiting time in queue}\brak{W_q}=\frac{L_q}{\lambda}$ & $\frac{5}{6}$hr\\[1ex]
    \hline     
    \end{tabular}
}
    \caption{Parameters of the given question and values.}
    \label{me2011-19:TABLE-1}
\end{table}