Let $X_1,X_2,X_3,....,X_n$ be the random variable for n independent trials such that 
\begin{align}
X &= X_1 + X_2 + X_2 + X_3 + \cdots + X_n \nonumber \\ 
X &= \sum_{i=1}^n X_i \nonumber 
\end{align}
p = success (1) and 1 - p = failure (0) 
Expected Value for n trials : 
\begin{align}
E(X_i) &= X_i\cdot p_i \nonumber\\
E(X_i) &= 1\cdot p + 0\cdot (1-p)\nonumber\\
E(X_i) &= p
\end{align}
We know that,
\begin{align}
E(X) &= \sum_{i=1}^n E (X_i) \nonumber\\
E(X) &= np
 \end{align}
Mean of a binomial distribution for n independent trials is \textbf{np}.
Now,
\begin{align}
E(X_i^2) &= X_i^2\cdot p_i \nonumber\\
E(X_i^2) &= 1^2\cdot p + 0^2\cdot (1-p)\nonumber\\
E(X_i^2) &= p
\end{align}
For variance,
\begin{align}
    Var(X_i) &= E(X_i^2) - E(X_i)^2 \nonumber\\
    Var(X_i) &= p - p^2 
    \end{align}
We can add Var($X_i$)  to get Var(X)  as these are independent trials
\begin{align}
Var(X) &= \sum_{i=1}^n Var(X_i) \nonumber \\
Var(X) &= n(p - p^2)\nonumber\\
Var(X) &= np(1-p)
\end{align}
Variance of a binomial distribution for n independent trials is \textbf{np(1-p)}.
Hence, (4) is correct option.
