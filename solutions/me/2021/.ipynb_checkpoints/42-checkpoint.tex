In a Poisson process,
 \begin{align}
          \pr{X=x}&= e^{-\lambda \Delta t} \frac{(\lambda \Delta t)^{x}}{x\,!}
 \end{align}
If $\Delta t \rightarrow 0$ then probability of having only one Poisson job is  
  \begin{align}
         \pr{X=1}=\lambda \Delta t \label{me2021-42:singlejob-condition}
 \end{align}
 Some assumptions:\\
 In time interval $\Delta t$,
 \begin{itemize}
     \item Exactly one job is arrived  
     \item or Exactly one job is completed
     \item or Nothing happens
 \end{itemize}
Assumptions seem quite reasonable as $\Delta t$ is very small then the probability of occurrence of more than one poisson job is very low.\\
 For job arrival,
 \begin{itemize}
 \item It is distributed according to Poisson distribution.
     \item Its
 Rate parameter $\lambda $=12 jobs/hour.
 \item Using \eqref{me2021-42:singlejob-condition},Probability that a single job arrives in a small interval $\Delta t=\lambda\Delta t$.
 \end{itemize}
 For Job completions,
 \begin{itemize}
     \item  Job completion time is distributed exponentially with mean of 4 minutes 
     \item Then we can assume that no. of job completions are distributed as Poisson distribution with rate parameter $\mu$ = 15 jobs/hour
     \item Once again using \eqref{me2021-42:singlejob-condition},
 Probability that a single job will be completed in a small interval $\Delta t=\mu \Delta t$
 \end{itemize}
 Some notations,
 \begin{table}[h]
\begin{tabular}{|c|p{6cm}|}
\hline
\textbf{Parameter} & \textbf{Definition}                               \\ \hline
$\lambda$          & Poisson rate parameter for the arrival of jobs    \\ \hline
$\mu$              & Poisson rate parameter for the completion of jobs \\ \hline
$\lambda \Delta t$ & Probability that a single job arrives in a small interval $\Delta t$\\\hline 
$\mu \Delta t$ & Probability that a single job will be completed in a small interval $\Delta t$\\\hline 
$P_j(t)$             & probability of having j jobs at workstation at time t \\\hline
$\pi_j$            & steady probability of having j jobs at workstation\\\hline
\end{tabular}
\caption{Parameters and their definitions used in the problem}
\label{me2021-42:tab:parameters}
\end{table}
 \begin{itemize}
     \item  Initial no.of jobs at workstation is 0.
     \item Let $P_{j}(t)$ denote the probability of having $j$ jobs waiting at the workstation at the time $t$ for this initial case.
     \item After a long time,probability of having  j jobs becomes steady.
     \item Let us denote steady state probability of having j jobs as $\pi_j$.
 \end{itemize}
 Condition which ensures that steady state is reached is
 \begin{align}
     \frac{\text{d}P_j(t)}{\text{d}t}&=0\\
     \lim_{\Delta t\rightarrow 0}\dfrac{P_j(t+\Delta t)-P_j (t)}{\Delta t}&=0\label{me2021-42:steady-condition}
 \end{align}
  We can reach a state of $j$ jobs at time $t+\Delta t$ from
  \begin{itemize}
      \item A state of $j-1$ jobs at time $t$ with a new job arriving in the next $\Delta t$
      \item A state of $j+1$ jobs at time $t$ with a job completing in the next $\Delta t$
      \item A state of $j$ jobs at time $t$ and nothing happening in the next $\Delta t$
  \end{itemize}
 Assuming time  $t$ is long enough for the occurrence of steady state.The above relations can be shown in probability equations as:  
 \begin{align}
     P_j(t+\Delta t)&=P_{j-1}(t)\lambda\Delta t+ P_{j+1}(t)\mu \Delta t \nonumber\\&+P_j (t) (1-\lambda\Delta t -\mu \Delta t)\\
     \dfrac{P_j(t+\Delta t)-P_j (t)}{\Delta t}&= P_{j-1}(t)\lambda +P_{j+1}(t)\mu\nonumber\\& - P_j(t)\lambda -P_j(t)\mu
\end{align}
Using \eqref{me2021-42:steady-condition} we get,
\begin{align}
     \implies P_{j-1}(t)\lambda +P_{j+1}(t)\mu&=P_j(t)\lambda +P_j(t)\mu \\
     \pi_{j-1}\lambda +\pi_{j+1}\mu&=\pi_j\lambda +\pi_j\mu\label{me2021-42:recursive} 
 \end{align}
 Note that the above equations are  for $j \geq 1$. \\
 For j=0 jobs at time $t+\Delta t$ we can reach it from j=1 job at time $t$ with a job completion in the next $\Delta t$ or else stay at j=0 at time $t$ and do nothing the next $\Delta t$
\begin{align}
    P_0(t+\Delta t)&=P_1(t)\mu \Delta t+\nonumber\\&~~~~P_0(t)(1-\lambda \Delta t)\\
    \dfrac{P_0(t+\Delta t)-P_0(t)}{\Delta t}&=P_1(t) \mu \Delta t-P_0(t)\lambda \Delta t
\end{align}
Once again using \eqref{me2021-42:steady-condition},we will get,
\begin{align}
    P_0(t)\lambda \Delta t&= P_1(t) \mu \Delta t\\
    P_0(t)\lambda&=P_1(t) \mu \\
    \pi_0 \lambda&=\pi_1\mu\label{me2021-42:base}
\end{align}
Solving \eqref{me2021-42:base} and \eqref{me2021-42:recursive} with appropriate j one by one,we will get $P_j$ in terms of $P_0$ as
\begin{equation}
    P_j=\brak{\dfrac{\lambda}{\mu}}^jP_0 
\end{equation}
consider $\rho = \dfrac{\lambda}{\mu}$.
\begin{table}[h]
\begin{tabular}{|c|p{6cm}|}
\hline
\textbf{Parameter} & \textbf{Definition}                               \\ \hline
$E(j)$             & Expected no. of jobs at workstation \\ \hline
$\rho$             & $\dfrac{\lambda}{\mu}$\\\hline
\end{tabular}
\caption{Parameters and their definitions used in the problem}
\label{me2021-42:tab:parameters2}
\end{table}
\begin{equation}
    P_j=\rho^j P_0 \label{me2021-42:solution}
\end{equation}
We can prove that \eqref{me2021-42:solution} is indeed the solution of recursion equation \eqref{me2021-42:recursive} by using mathematical induction.\\
Assuming $\rho<1$,let us calculate $P_0$ in terms of $\rho$
\begin{align}
    \sum_{j=0}^{\infty}P_j&=1\\
    \sum_{j=0}^{\infty}\rho^j P_0 &=1\\
    \dfrac{P_0}{1-\rho}&=1\\
    P_0&=1-\rho
\end{align}
This yields,\\
\begin{equation}
    P_j=\rho^j(1-\rho)
\end{equation}
Let us calculate expected value of jobs waiting at workstation.
\begin{align}
    E(j)&=\sum_{j=0}^{\infty}jP_j\\
    E(j)&=(1-\rho)\sum_{j=0}^{\infty}j\rho^j\label{me2021-42:first}\\
    \rho E(j)&=(1-\rho)\sum_{j=0}^{\infty}j\rho^{j+1}\\
    \rho E(j)&=(1-\rho)\sum_{j=1}^{\infty}(j-1)\rho^{j}\label{me2021-42:second}
\end{align}
Subtracting \eqref{me2021-42:second} from \eqref{me2021-42:first}.we get,
\begin{align}
    (1-\rho)E(j)&=(1-\rho)\sum_{j=1}^{\infty}\rho^j\\
    E(j)&=\sum_{j=1}^{\infty}\rho^j\\
    E(j)&=\dfrac{\rho}{1-\rho}\label{me2021-42:expect}
\end{align}
In our case $\rho=\dfrac{\lambda}{\mu}=\dfrac{12}{15}=\dfrac{4}{5}$. Substituting it in the \eqref{me2021-42:expect} we get,\\
\begin{equation}
    E(j)=4
\end{equation}
$\therefore$ Expected no.of jobs at workstation is 4.