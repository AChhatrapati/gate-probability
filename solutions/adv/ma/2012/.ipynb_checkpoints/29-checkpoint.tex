%
  Given that random variable X  assumes only positive integral values and its probability is:
  \begin{align}
      P(X=x)=\frac{2}{3}\brak{\frac{1}{3}}^{x-1}
  \end{align}
  The expectation value E(X) is given by 
  \begin{align}
      E(X)=\sum_{i=1}^{\infty} i \times P(X=i)
  \end{align}
  Let $E(X)=S$\\
  so,
  \begin{align}
      S&=\sum_{i=1}^{\infty} i \times P(X=i)\\
    \implies S&=\sum_{i=1}^{\infty} i \times
      \frac{2}{3}\brak{\frac{1}{3}}^{i-1}\label{ma2012-29:eq:0.0.5}\\
     \implies S&=\frac{2}{3}+\sum_{i=2}^{\infty} i \times
      \frac{2}{3}\brak{\frac{1}{3}}^{i-1}\label{ma2012-29:eq:0.0.6}
  \end{align}
  As
  \begin{align}
      \sum_{i=2}^{\infty} i \times
      \frac{2}{3}\brak{\frac{1}{3}}^{i-1}=\sum_{i=1}^{\infty} (i+1)\times
      \frac{2}{3}\brak{\frac{1}{3}}^{i}\label{ma2012-29:eq:0.0.7}
  \end{align}
  Now substituting  \eqref{ma2012-29:eq:0.0.7} in \eqref{ma2012-29:eq:0.0.6}
  \begin{align}
      \implies S&=\frac{2}{3}+\sum_{i=1}^{\infty} (i+1) \times
      \frac{2}{3}\brak{\frac{1}{3}}^{i}
  \end{align}
  \begin{align}
       \implies S&=\frac{2}{3}+\sum_{i=1}^{\infty} i \times
      \frac{2}{3}\brak{\frac{1}{3}}^{i}+\sum_{i=1}^{\infty} 
      \frac{2}{3}\brak{\frac{1}{3}}^{i}\label{ma2012-29:eq:0.0.9}
  \end{align}
   Dividing with 3 on both sides in \eqref{ma2012-29:eq:0.0.5} gives
   \begin{align}
      \frac{S}{3}=\sum_{i=1}^{\infty} i \times
      \frac{2}{3}\brak{\frac{1}{3}}^{i}\label{ma2012-29:eq:0.0.10}
  \end{align}
  Now substituting \eqref{ma2012-29:eq:0.0.10} in \eqref{ma2012-29:eq:0.0.9} gives
  \begin{align}
     \implies S&=\frac{2}{3}+\frac{S}{3}+\sum_{i=1}^{\infty}\frac{2}{3}\brak{\frac{1}{3}}^{i}
     \end{align}
     \begin{align}
     \implies \frac{2S}{3}&=\frac{2}{3}+\frac{2}{3}\sum_{i=1}^{\infty}\brak{\frac{1}{3}}^{i}
  \end{align}
  \begin{align}
     \implies \frac{2S}{3}&=\frac{2}{3}\brak{
     1+\sum_{i=1}^{\infty}\brak{\frac{1}{3}}^{i}}
  \end{align}
  \begin{align}
     \implies S=1+\sum_{i=1}^{\infty}\brak{\frac{1}{3}}^{i}
  \end{align}
  \begin{align}
     \implies S=1+\frac{\frac{1}{3}}{1-\frac{1}{3}}
  \end{align}
  \begin{align}
     \implies S=1+\frac{1}{2}=\frac{3}{2}
  \end{align}
  \begin{align}
     \implies E(X)=S=\frac{3}{2}
  \end{align}
  %
  \textbf{$\therefore$ Option D is correct}    