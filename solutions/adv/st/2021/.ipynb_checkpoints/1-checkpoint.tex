
Given, for X $>$ 0 ,$E(X) = 9$, $E\brak{\frac{1}{X+1}}$ can be estimated by Jensens's Inequality. \\
\textbf{pre - requisites:}\\
In general, $\phi(X)$ is a convex function iff:
\begin{equation*}
    \frac{d^2 \phi}{dX^2} \ge 0
\end{equation*}
\textbf{Jensen's Inequality:}\\
In the context of probability theory, it is generally stated in the following form: if X is a random variable and $\phi$ is a convex function, then
\begin{equation*}
\tag{1} \label{st2021-1:jenson}
    \phi(E(X)) \le E(\phi(X))
\end{equation*}
\begin{align*}
    \text{So for } \phi(X) &= \frac{1}{X+1}, \\
                \frac{d\phi}{dX} &= - \frac{1}{(X+1)^{2}} \\
                \tag{2} \label{st2021-1:phi}
                \frac{d^2 \phi}{dX^2} &= \frac{2}{(X+1)^{3}} 
    \implies \frac{d^2 \phi}{dX^2} \ge 0,(\because X>0 )
\end{align*}
by eq \eqref{st2021-1:jenson} and \eqref{st2021-1:phi}
\begin{align*}
    E\brak{\frac{1}{X+1}} &\ge \frac{1}{E(X)+1} \\
    \implies E\brak{\frac{1}{X+1}} &\ge \frac{1}{9 + 1} \\
    \tag{3} \label{st2021-1:Part 1}
    \implies E\brak{\frac{1}{X+1}} &\ge 0.1
\end{align*}
$\pr{X \ge 10}$ can be estimated by Markov's Inequality.\\
\textbf{Markov's Inequality:}
If X is a non-negative random variable and a $>$ 0, then the probability that X is at least a is at most the expectation of X divided by a. \\
Mathematically,
\begin{equation*}
    \tag{4} \label{st2021-1:markov}
    \pr{X \ge a} \le \frac{E(X)}{a}
\end{equation*}
by \eqref{st2021-1:markov} for a = 10
\begin{align*}
    \pr{X \ge 10} &\le \frac{E(X)}{10} \\
    \implies \pr{X \ge 10} &\le \frac{9}{10} \\
    \tag{5} \label{st2021-1:Part 2}
   \therefore \pr{X \ge 10} &\le 0.9
\end{align*}
So, from \eqref{st2021-1:Part 1} and \eqref{st2021-1:Part 2} \\
\begin{center}
     \boxed{\textbf{Option 1 is the Correct Answer}}
\end{center}