

Given,
\begin{align}
\pr{B}=\frac{3}{4}\label{st2021-14:a}\\
\pr{A + B^{\prime}} =\frac{1}{2} \label{st2021-14:b}
\end{align}
we know that,
\begin{align}
\pr{B^{\prime}} = 1-\pr{B} \label{st2021-14:c}
\end{align}
using \eqref{st2021-14:a} in \eqref{st2021-14:c},
\begin{align}
\pr{B^{\prime}} = \frac{1}{4} \label{st2021-14:d}
\end{align}
we know that,
\begin{align}
\pr{A + B^{\prime}} = \pr{A} +\pr{B^{\prime}} - \pr{A,B^{\prime}}
\end{align}
A and B are independent $\iff$ A and $B^{\prime}$ are independent
\begin{align}
\pr{A + B^{\prime}} = \pr{A} +\pr{B^{\prime}} - \pr{A}\pr{B^{\prime}} \label{st2021-14:e}
\end{align}
using \eqref{st2021-14:b}  and \eqref{st2021-14:d} in \eqref{st2021-14:e},
\begin{align}
\frac{1}{2} &= \pr{A} + \frac{1}{4} - \frac{\pr{A}}{4}\\
\frac{1}{4} &= \frac{3\pr{A}}{4}\\
\therefore \pr{A} &= \frac{1}{3}
\end{align}