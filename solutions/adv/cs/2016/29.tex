Given a fair coin is flipped twice.\\
Let us define a Markov chain with states \{1,2,3,4\}, such that 
\begin{table}[h!]
\resizebox{\columnwidth}{!}{
\begin{tabular}{|c|l|}
\hline
\textbf{State} & \multicolumn{1}{c|}{\textbf{Events}}                     \\ \hline
1                 & Event of tossing a fair coin twice                           \\ \hline
2                 & Event of obtaining 'N' as the output                         \\ \hline
3                 & Event of obtaining 'Y' as the output                         \\ \hline
4                 & Event of obtaining (TAIL,TAIL) as the output                 \\ \hline
\end{tabular}
}
\caption{Representation of different events}
\label{cs/2016/29table:1}
\end{table}
\\We know that when a fair coin is tossed,
\begin{align}
    \pr{\text{HEAD}} &=  1/2 \text{ and }\\
    \pr{\text{TAIL}} &=  1/2.
\end{align}
Then, \\
\begin{figure}
    \centering
    \caption{\textbf{Markov chain diagram}}
    \label{cs/2016/29fig:1}
    \begin{tikzpicture}[font=\sffamily]
        % Setup the style for the states
        \tikzset{node style/.style={state, 
                                    minimum width=2cm,
                                    line width=0.5mm,
                                    fill=gray!20!white}}
        % Draw the states
        \node[node style] at (0, 0)     (2)     {2};
        \node[node style] at (3, 0)     (3)     {3};
        \node[node style] at (6, 0)     (4)     {4};
        \node[node style] at (3, -5)    (1)     {1};
        
        % Connect the states with arrows
        \draw[every loop,
              auto=right,
              line width=1mm,
              >=latex,
              draw=orange,
              fill=orange]
            (1)     edge[bend left=10, auto=left]  node {0.5} (2)
            (1)     edge                node {0.25} (3)
            (1)     edge[bend right=10, auto=right] node {0.25} (4)
            (4)     edge[bend left=50, auto=left]            node {1} (1)
            (3)     edge[loop above]                node {1} (3)
            (2)     edge[loop above]                node {1} (2)
            ;
\end{tikzpicture}
\end{figure}
The state transition matrix (P) for the Markov chain is
\begin{align}
\label{cs/2016/29eq:1}
    P=\begin{blockarray}{ccccc}
& 1 & 2 & 3 & 4 \\
\begin{block}{c[cccc]}
  1 & 0 & \frac{1}{2} & \frac{1}{4}  & \frac{1}{4}  \\
  2 & 0 & 1 & 0 & 0 \\
  3 & 0 & 0 & 1 & 0 \\
  4 & 1 & 0 & 0 & 0 \\
\end{block}
\end{blockarray}
\end{align}
By the definition of transient and absorbing states, we can say that 1,4 are transient states whereas 2,3 are absorbing.\\
Then, the canonical form of the transition matrix is,
\begin{align}
\label{cs/2016/29eq:2}
    P=\begin{blockarray}{ccccc}
& 2 & 3 & 1 & 4 \\
\begin{block}{c[cc|cc]}
  2 & 1 & 0 & 0 & 0  \\
  3 & 0 & 1 & 0 & 0 \\ 
  \cline{2-5}
  1 & \frac{1}{2} & \frac{1}{4} & 0 & \frac{1}{4} \\
  4 & 0 & 0 & 1 & 0 \\
\end{block}
\end{blockarray}
\end{align}
The canonical form divides the transition matrix into four sub-matrices based on the states as listed below.\\
\begin{blockarray}{ccc}
& \text{Absorbing} & \text{Non-Absorbing} \\
\begin{block}{c[c|c]}
  \text{Absorbing} & I & O \\
  \cline{2-3}
  \text{Non-Absorbing} & A & B \\
  \end{block}
\end{blockarray}
\\where,
\begin{table}[h!]
\centering
\begin{tabular}{|c|c|}
    \hline
    \textbf{Variable} & \textbf{Type of Matrix} \\
    \hline
    $I$ & Identity matrix\\
    \hline
    $O$ & Zero matrix\\
    \hline
    $A,B$ & Some matrices\\
    \hline
\end{tabular}
\caption{Representation of different matrices}
\label{cs/2016/29table:2}
\end{table}
\\and From \eqref{cs/2016/29eq:2},
\begin{align}
\label{cs/2016/29eq:3}
    A=\begin{bmatrix}
    \frac{1}{2} & \frac{1}{4}\\
    0 & 0\\
    \end{bmatrix},
    B=\begin{bmatrix}
    0 & \frac{1}{4} \\
    1 & 0 \\
    \end{bmatrix}
\end{align}
The fundamental matrix F for the absorbing Markov chain is defined as 
\begin{align}
    F &=(I-B)^{-1} \label{cs/2016/29eq:4}\\
\shortintertext{Then,  }
    F &= \begin{bmatrix}
    1 & -\frac{1}{4} \\
    -1 & 1 \\ 
    \end{bmatrix}^{-1} \\
    \implies F &= \begin{bmatrix}
               1.33 & 0.33 \\
               1.33 & 1.33 \\ 
               \end{bmatrix}\\
\shortintertext{Therefore,}
    FA &= \begin{bmatrix}
    0.67 & 0.33 \\
    0.67 & 0.33 \\ 
    \end{bmatrix}
    \end{align}
Then the limiting matrix for the markov chain is 
\begin{align}
\label{cs/2016/29eq:5}
    \bar P&=\begin{bmatrix}
    I & O\\
    FA & O\\
    \end{bmatrix}
\end{align}
where the element $p_{ij}$ of $\bar P$ represents the probability of absorption in state j, when the initial state is i.
\begin{align}
  \therefore \bar P&=\begin{blockarray}{ccccc}
                  & 2 & 3 & 1 & 4 \\
                  \begin{block}{c[cccc]}
                  2 & 1 & 0 & 0 & 0  \\
                  3 & 0 & 1 & 0 & 0 \\ 
                  1 & 0.67 & 0.33 & 0 & 0 \\
                  4 & 0.67 & 0.33 & 0 & 0 \\
                \end{block}
                \end{blockarray}  
\end{align}
Therefore,
\begin{align}
    \text{ Req. Probability }= p_{13} = 0.33
\end{align}