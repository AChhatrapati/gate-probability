Given,\\
Set A= [2,3,4,5]\\
Set B= [11,12,13,14,15]\\

Total number of element in the sample space is 20.\\

Let us define a random variable$ X \in \cbrak{0,1}$\\
\begin{table}[ht]
    \centering
    \begin{tabular}{|c|c|}
    \hline
    X=0 & the event when A+B=16\\
    \hline
    X=1 & the event when A+B $\neq$ 16\\
    \hline
    \end{tabular}
    \caption{Random Variables}
    \label{tab:Random Variables}
\end{table}

Now, probability of selecting an element from set A such that \pr{X=0} is
\begin{equation}
   \tag{7.1}
    \pr{X=0}=\pr{A+B=16}=1
\end{equation}

So,the probability of selecting an element from set B after selecting an element from set A such that \pr{X=0} is
\begin{equation}
    \tag{7.2}
    \pr{X=0}=\pr{A+B=16}= \frac{1}{5}
\end{equation}

Therefore,\\
Overall probability of randomly choosing elements from set A and set B such that \pr{X=0}is
\begin{align}
    \tag{7.3}
    \pr{X=0}&=\pr{A+B=16}\\
    \tag{7.4}
    \pr{X=0}&=1 \times \frac{1}{5}\\
    \tag{7.5}
    \pr{X=0}&= \frac{1}{5}=0.2
\end{align}


\begin{table}[ht]
    \centering
    \begin{tabular}{|c|c|c|}
    \hline
    X & 0 & 1\\
    \hline
    $\pr{X}$ & $\frac{1}{5}$ & $\frac{4}{5}$\\
    \hline
    \end{tabular}
    \caption{Probability distribution table}
    \label{tab:Probability distribution table}
\end{table}

Therefore, the correct option is (a).