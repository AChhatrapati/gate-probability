\begin{enumerate}[label=\thechapter.\arabic*,ref=\thechapter.\theenumi]
\item Suppose that \(x\) is an observed sample of size 1 from a population with probability density function \(f(\cdot)\). Based on \(x\), consider testing
\[
H_0: f(y) = \frac{1}{\sqrt{2\pi}} e^{-\frac{y^2}{2}};\quad y \in \mathbb{R}
\]
against
\[
H_1: f(y) = \frac{1}{2} e^{-|y|};\quad y \in \mathbb{R}.
\]
Then which one of the following statements is true?
\begin{enumerate}
    \item The most powerful test rejects $H_0$ if $|x| > c$ for some $c > 0$ \label{eq:ST/28/2023/Option1}
    \item The most powerful test rejects $H_0$ if $|x| < c$ for some $c > 0$ \label{eq:ST/28/2023/Option2}
    \item The most powerful test rejects $H_0$ if $| |x| - 1| > c$ for some $c > 0$ \label{eq:ST/28/2023/Option3}
    \item The most powerful test rejects $H_0$ if $| |x| - 1| < c$ for some $c > 0$ \label{eq:ST/28/2023/Option4}
\end{enumerate} 
\hfill (GATE ST 2023)
\iffalse
\let\negmedspace\undefined
\let\negthickspace\undefined
\documentclass[journal,12pt,twocolumn]{IEEEtran}
\usepackage{setspace}
\singlespacing
\usepackage[cmex10]{amsmath}
\usepackage{amsthm}
\usepackage{mathrsfs}
\usepackage{txfonts}
\usepackage{stfloats}
\usepackage{bm}
\usepackage{cite}
\usepackage{cases}
\usepackage{subfig}
\usepackage{longtable}
\usepackage{multirow}
\usepackage{enumitem}
\usepackage{mathtools}
\usepackage{tikz}
\usepackage{circuitikz}
\usepackage{verbatim}
\usepackage[breaklinks=true]{hyperref}
\usepackage{tkz-euclide} % loads  TikZ and tkz-base
\usepackage{listings}
\usepackage{color}    
\usepackage{array}    
\usepackage{longtable}
\usepackage{calc}     
\usepackage{multirow} 
\usepackage{hhline}   
\usepackage{ifthen}   
\usepackage{lscape}     
\usepackage{chngcntr}
\DeclareMathOperator*{\Res}{Res}
\renewcommand\thesection{\arabic{section}}
\renewcommand\thesubsection{\thesection.\arabic{subsection}}
\renewcommand\thesubsubsection{\thesubsection.\arabic{subsubsection}}
\renewcommand\thesectiondis{\arabic{section}}
\renewcommand\thesubsectiondis{\thesectiondis.\arabic{subsection}}
\renewcommand\thesubsubsectiondis{\thesubsectiondis.\arabic{subsubsection}}
\renewcommand{\thefigure}{\theenumi}
\renewcommand{\thetable}{\theenumi}
\providecommand{\gauss}[2]{\mathcal{N}\ensuremath{\left(#1,#2\right)}}
% correct bad hyphenation here
\hyphenation{op-tical net-works semi-conduc-tor}
\def\inputGnumericTable{}                                 %%

\lstset{
%language=C,
frame=single, 
breaklines=true,
columns=fullflexible
}
%\lstset{
%language=tex,
%frame=single, 
%breaklines=true
%}
\begin{document}
\newtheorem{theorem}{Theorem}[section]
\newtheorem{problem}{Problem}
\newtheorem{proposition}{Proposition}[section]
\newtheorem{lemma}{Lemma}[section]
\newtheorem{corollary}[theorem]{Corollary}
\newtheorem{example}{Example}[section]
\newtheorem{definition}[problem]{Definition}
\newcommand{\BEQA}{\begin{eqnarray}}
\newcommand{\EEQA}{\end{eqnarray}}
\newcommand{\define}{\stackrel{\triangle}{=}}
\bibliographystyle{IEEEtran}
\providecommand{\mbf}{\mathbf}
\providecommand{\pr}[1]{\ensuremath{\Pr\left(#1\right)}}
\providecommand{\qfunc}[1]{\ensuremath{Q\left(#1\right)}}
\providecommand{\sbrak}[1]{\ensuremath{{}\left[#1\right]}}
\providecommand{\lsbrak}[1]{\ensuremath{{}\left[#1\right.}}
\providecommand{\rsbrak}[1]{\ensuremath{{}\left.#1\right]}}
\providecommand{\brak}[1]{\ensuremath{\left(#1\right)}}
\providecommand{\lbrak}[1]{\ensuremath{\left(#1\right.}}
\providecommand{\rbrak}[1]{\ensuremath{\left.#1\right)}}
\providecommand{\cbrak}[1]{\ensuremath{\left\{#1\right\}}}
\providecommand{\lcbrak}[1]{\ensuremath{\left\{#1\right.}}
\providecommand{\rcbrak}[1]{\ensuremath{\left.#1\right\}}}
\theoremstyle{remark}
\newtheorem{rem}{Remark}
\newcommand{\sgn}{\mathop{\mathrm{sgn}}}
\providecommand{\abs}[1]{\left\vert#1\right\vert}
\providecommand{\res}[1]{\Res\displaylimits_{#1}} 
\providecommand{\norm}[1]{\left\lVert#1\right\rVert}
\providecommand{\mtx}[1]{\mathbf{#1}}
\providecommand{\mean}[1]{E\left[ #1 \right]}
\providecommand{\fourier}{\overset{\mathcal{F}}{ \rightleftharpoons}}
\providecommand{\system}[1]{\overset{\mathcal{#1}}{ \longleftrightarrow}}
\newcommand{\solution}{\noindent \textbf{Solution: }}
\newcommand{\cosec}{\,\text{cosec}\,}
\providecommand{\dec}[2]{\ensuremath{\overset{#1}{\underset{#2}{\gtrless}}}}
\newcommand{\myvec}[1]{\ensuremath{\begin{pmatrix}#1\end{pmatrix}}}
\newcommand{\mydet}[1]{\ensuremath{\begin{vmatrix}#1\end{vmatrix}}}
\let\vec\mathbf
\def\putbox#1#2#3{\makebox[0in][l]{\makebox[#1][l]{}\raisebox{\baselineskip}[0in][0in]{\raisebox{#2}[0in][0in]{#3}}}}
     \def\rightbox#1{\makebox[0in][r]{#1}}
     \def\centbox#1{\makebox[0in]{#1}}
     \def\topbox#1{\raisebox{-\baselineskip}[0in][0in]{#1}}
     \def\midbox#1{\raisebox{-0.5\baselineskip}[0in][0in]{#1}}
\setlength{\parindent}{0pt}
\bibliographystyle{IEEEtran}
\newenvironment{amatrix}[1]{%
  \left(\begin{array}{@{}*{#1}{c}|c@{}}
}{%
  \end{array}\right)
}
\title{
%	\logo{
Probability and Random Processes
%	}
}
\author{ Gude Pravarsh EE22BTECH11023$^{*}$% <-this % stops a space
	%
}
	
	
%\title{
%	\logo{Matrix Analysis through Octave}{\begin{center}\includegraphics[scale=.24]{tlc}\end{center}}{}{HAMDSP}
%}


% paper title
% can use linebreaks \\ within to get better formatting as desired
%\title{Matrix Analysis through Octave}
%
%
% author names and IEEE memberships
% note positions of commas and nonbreaking spaces ( ~ ) LaTeX will not break
% a structure at a ~ so this keeps an author's name from being broken across
% two lines.
% use \thanks{} to gain access to the first footnote area
% a separate \thanks must be used for each paragraph as LaTeX2e's \thanks
% was not built to handle multiple paragraphs
%
%\author{<-this % stops a space
%\thanks{}}
%}
% note the % following the last \IEEEmembership and also \thanks - 
% these prevent an unwanted space from occurring between the last author name
% and the end of the author line. i.e., if you had this:
% 
% \author{....lastname \thanks{...} \thanks{...} }
%                     ^------------^------------^----Do not want these spaces!
%
% a space would be appended to the last name and could cause every name on that
% line to be shifted left slightly. This is one of those "LaTeX things". For
% instance, "\textbf{A} \textbf{B}" will typeset as "A B" not "AB". To get
% "AB" then you have to do: "\textbf{A}\textbf{B}"
% \thanks is no different in this regard, so shield the last } of each \thanks
% that ends a line with a % and do not let a space in before the next \thanks.
% Spaces after \IEEEmembership other than the last one are OK (and needed) as
% you are supposed to have spaces between the names. For what it is worth,
% this is a minor point as most people would not even notice if the said evil
% space somehow managed to creep in.



% The paper headers
%\markboth{Journal of \LaTeX\ Class Files,~Vol.~6, No.~1, January~2007}%
%{Shell \MakeLowercase{\textit{et al.}}: Bare Demo of IEEEtran.cls for Journals}
% The only time the second header will appear is for the odd numbered pages
% after the title page when using the twoside option.
% 
% *** Note that you probably will NOT want to include the author's ***
% *** name in the headers of peer review papers.                   ***
% You can use \ifCLASSOPTIONpeerreview for conditional compilation here if
% you desire.




% If you want to put a publisher's ID mark on the page you can do it like
% this:
%\IEEEpubid{0000--0000/00\$00.00~\copyright~2007 IEEE}
% Remember, if you use this you must call \IEEEpubidadjcol in the second
% column for its text to clear the IEEEpubid mark.
% make the title area 
\maketitle

\newpage

%\tableofcontents

\bigskip

\renewcommand{\thefigure}{\arabic{figure}}
\renewcommand{\thetable}{\theenumi}
%\renewcommand{\theequation}{\theenumi}

%\begin{abstract}
%%\boldmath
%In this letter, an algorithm for evaluating the exact analytical bit error rate  (BER)  for the piecewise linear (PL) combiner for  multiple relays is presented. Previous results were available only for upto three relays. The algorithm is unique in the sense that  the actual mathematical expressions, that are prohibitively large, need not be explicitly obtained. The diversity gain due to multiple relays is shown through plots of the analytical BER, well supported by simulations. 
%
%\end{abstract}
% IEEEtran.cls defaults to using nonbold math in the Abstract.
% This preserves the distinction between vectors and scalars. However,
% if the journal you are submitting to favors bold math in the abstract,
% then you can use LaTeX's standard command \boldmath at the very start
% of the abstract to achieve this. Many IEEE journals frown on math
% in the abstract anyway.

% Note that keywords are not normally used for peerreview papers.
%\begin{IEEEkeywords}
%Cooperative diversity, decode and forward, piecewise linear
%\end{IEEEkeywords} 
Q)Suppose that \(x\) is an observed sample of size 1 from a population with probability density function \(f(\cdot)\). Based on \(x\), consider testing
\[
H_0: f(y) = \frac{1}{\sqrt{2\pi}} e^{-\frac{y^2}{2}};\quad y \in \mathbb{R}
\]
against
\[
H_1: f(y) = \frac{1}{2} e^{-|y|};\quad y \in \mathbb{R}.
\]
Then which one of the following statements is true?
\begin{enumerate}
    \item The most powerful test rejects $H_0$ if $|x| > c$ for some $c > 0$ \label{eq:ST/28/2023/Option1}
    \item The most powerful test rejects $H_0$ if $|x| < c$ for some $c > 0$ \label{eq:ST/28/2023/Option2}
    \item The most powerful test rejects $H_0$ if $| |x| - 1| > c$ for some $c > 0$ \label{eq:ST/28/2023/Option3}
    \item The most powerful test rejects $H_0$ if $| |x| - 1| < c$ for some $c > 0$ \label{eq:ST/28/2023/Option4}
\end{enumerate} \hfill ( GATE ST 2023) \\
   \fi
   \solution 
\begin{align}
L = \prod\limits_{i=1}^{1} f(x) = f(x)
\end{align}
To determine the most powerful test, we need to consider the likelihood ratio test
\begin{align}
\frac{L(H_1)}{L(H_0)} \overset{H_1}{\underset{H_0}{\gtrless}} k \\
\implies \frac{\frac{1}{\sqrt{2\pi}} e^{-\frac{x^2}{2}}}{\frac{1}{2} e^{-2|x|}} \overset{H_1}{\underset{H_0}{\gtrless}} k \\
\implies e^{\frac{x^2-2|x|}{2}} \overset{H_1}{\underset{H_0}{\gtrless}} k\frac{\sqrt{\pi}}{\sqrt{2}} \\ 
(|x| - 1)^2 \overset{H_1}{\underset{H_0}{\gtrless}} 2\log\left(\frac{k\sqrt{\pi}}{\sqrt{2}}\right) + 1 
\end{align}
Taking square root on both sides,
\begin{align}
 ||x| - 1| \overset{H_1}{\underset{H_0}{\gtrless}} \sqrt{2\log\left(\frac{k\sqrt{\pi}}{\sqrt{2}}\right) + 1} \\
\implies \lvert x \rvert \overset{H_1}{\underset{H_0}{\gtrless}} 1 + \sqrt{2\log\left(\frac{k\sqrt{\pi}}{\sqrt{2}}\right) + 1}
\end{align}
Hence, the correct answer is \eqref{eq:ST/28/2023/Option3}

\item Suppose that $X_1, X_2, \ldots, X_n$ are independent and identically distributed random variables, each having probability density function $f(\cdot)$ and median $\theta$. We want to test\\
$H_0: \theta = \theta_0$ against $H_1: \theta > \theta_0$\\
Consider a test that rejects $H_0$ if $S > c$ for some $c$ depending on the size of the test, where $S$ is the cardinality of the set $\{i : X_i > \theta_0, 1 \leq i \leq n\}$. Then which one of the following statements is true?
\begin{enumerate}
\item Under $H_0$, the distribution of $S$ depends on $f(\cdot)$.
\item Under $H_1$, the distribution of $S$ does not depend on $f(\cdot)$.
\item The power function depends on $\theta$.
\item The power function does not depend on $\theta$.
\end{enumerate}
\hfill (GATE ST 2023)
\iffalse
\let\negmedspace\undefined
\let\negthickspace\undefined
\documentclass[journal,12pt,onecolumn]{IEEEtran}
\usepackage{cite}
\usepackage{amsmath,amssymb,amsfonts,amsthm}
\usepackage{algorithmic}
\usepackage{graphicx}
\usepackage{textcomp}
\usepackage{xcolor}
\usepackage{txfonts}
\usepackage{listings}
\usepackage{enumitem}
\usepackage{mathtools}
\usepackage{gensymb}
\usepackage[breaklinks=true]{hyperref}
\usepackage{tkz-euclide} % loads  TikZ and tkz-base
\usepackage{listings}



\newtheorem{theorem}{Theorem}[section]
\newtheorem{problem}{Problem}
\newtheorem{proposition}{Proposition}[section]
\newtheorem{lemma}{Lemma}[section]
\newtheorem{corollary}[theorem]{Corollary}
\newtheorem{example}{Example}[section]
\newtheorem{definition}[problem]{Definition}
%\newtheorem{thm}{Theorem}[section] 
%\newtheorem{defn}[thm]{Definition}
%\newtheorem{algorithm}{Algorithm}[section]
%\newtheorem{cor}{Corollary}
\newcommand{\BEQA}{\begin{eqnarray}}
\newcommand{\EEQA}{\end{eqnarray}}
\newcommand{\define}{\stackrel{\triangle}{=}}
\theoremstyle{remark}
\newtheorem{rem}{Remark}
%\bibliographystyle{ieeetr}
\begin{document}
%
\providecommand{\pr}[1]{\ensuremath{\Pr\left(#1\right)}}
\providecommand{\prt}[2]{\ensuremath{p_{#1}^{\left(#2\right)} }}        % own macro for this question
\providecommand{\qfunc}[1]{\ensuremath{Q\left(#1\right)}}
\providecommand{\sbrak}[1]{\ensuremath{{}\left[#1\right]}}
\providecommand{\lsbrak}[1]{\ensuremath{{}\left[#1\right.}}
\providecommand{\rsbrak}[1]{\ensuremath{{}\left.#1\right]}}
\providecommand{\brak}[1]{\ensuremath{\left(#1\right)}}
\providecommand{\lbrak}[1]{\ensuremath{\left(#1\right.}}
\providecommand{\rbrak}[1]{\ensuremath{\left.#1\right)}}
\providecommand{\cbrak}[1]{\ensuremath{\left\{#1\right\}}}
\providecommand{\lcbrak}[1]{\ensuremath{\left\{#1\right.}}
\providecommand{\rcbrak}[1]{\ensuremath{\left.#1\right\}}}
\newcommand{\sgn}{\mathop{\mathrm{sgn}}}
\providecommand{\abs}[1]{\left\vert#1\right\vert}
\providecommand{\res}[1]{\Res\displaylimits_{#1}} 
\providecommand{\norm}[1]{\left\lVert#1\right\rVert}
%\providecommand{\norm}[1]{\lVert#1\rVert}
\providecommand{\mtx}[1]{\mathbf{#1}}
\providecommand{\mean}[1]{E\left[ #1 \right]}
\providecommand{\cond}[2]{#1\middle|#2}
\providecommand{\fourier}{\overset{\mathcal{F}}{ \rightleftharpoons}}
\newenvironment{amatrix}[1]{%
  \left(\begin{array}{@{}*{#1}{c}|c@{}}
}{%
  \end{array}\right)
}
%\providecommand{\hilbert}{\overset{\mathcal{H}}{ \rightleftharpoons}}
%\providecommand{\system}{\overset{\mathcal{H}}{ \longleftrightarrow}}
	%\newcommand{\solution}[2]{\textbf{Solution:}{#1}}
\newcommand{\solution}{\noindent \textbf{Solution: }}
\newcommand{\cosec}{\,\text{cosec}\,}
\providecommand{\dec}[2]{\ensuremath{\overset{#1}{\underset{#2}{\gtrless}}}}
\newcommand{\myvec}[1]{\ensuremath{\begin{pmatrix}#1\end{pmatrix}}}
\newcommand{\mydet}[1]{\ensuremath{\begin{vmatrix}#1\end{vmatrix}}}
\newcommand{\myaugvec}[2]{\ensuremath{\begin{amatrix}{#1}#2\end{amatrix}}}
\providecommand{\rank}{\text{rank}}
\providecommand{\pr}[1]{\ensuremath{\Pr\left(#1\right)}}
\providecommand{\qfunc}[1]{\ensuremath{Q\left(#1\right)}}
	\newcommand*{\permcomb}[4][0mu]{{{}^{#3}\mkern#1#2_{#4}}}
\newcommand*{\perm}[1][-3mu]{\permcomb[#1]{P}}
\newcommand*{\comb}[1][-1mu]{\permcomb[#1]{C}}
\providecommand{\qfunc}[1]{\ensuremath{Q\left(#1\right)}}
\providecommand{\gauss}[2]{\mathcal{N}\ensuremath{\left(#1,#2\right)}}
\providecommand{\diff}[2]{\ensuremath{\frac{d{#1}}{d{#2}}}}
\providecommand{\myceil}[1]{\left \lceil #1 \right \rceil }
\newcommand\figref{Fig.~\ref}
\newcommand\tabref{Table~\ref}
\newcommand{\sinc}{\,\text{sinc}\,}
\newcommand{\rect}{\,\text{rect}\,}
%%
%	%\newcommand{\solution}[2]{\textbf{Solution:}{#1}}
%\newcommand{\solution}{\noindent \textbf{Solution: }}
%\newcommand{\cosec}{\,\text{cosec}\,}
%\numberwithin{equation}{section}
%\numberwithin{equation}{subsection}
%\numberwithin{problem}{section}
%\numberwithin{definition}{section}
%\makeatletter
%\@addtoreset{figure}{problem}
%\makeatother

%\let\StandardTheFigure\thefigure
\let\vec\mathbf

\bibliographystyle{IEEEtran}


\vspace{3cm}



\bigskip

\renewcommand{\thefigure}{\theenumi}
\renewcommand{\thetable}{\theenumi}
%\renewcommand{\theequation}{\theenumi}
Q: Suppose that $X_1, X_2, \ldots, X_n$ are independent and identically distributed random variables, each having probability density function $f(\cdot)$ and median $\theta$. We want to test\\
$H_0: \theta = \theta_0$ against $H_1: \theta > \theta_0$\\
Consider a test that rejects $H_0$ if $S > c$ for some $c$ depending on the size of the test, where $S$ is the cardinality of the set $\{i : X_i > \theta_0, 1 \leq i \leq n\}$. Then which one of the following statements is true?

(A) Under $H_0$, the distribution of $S$ depends on $f(\cdot)$.

(B) Under $H_1$, the distribution of $S$ does not depend on $f(\cdot)$.

(C) The power function depends on $\theta$.

(D) The power function does not depend on $\theta$.
\fi
\\ \solution

\begin{definition}
 Median \(\theta\) is defined as\\
\centering
\(\Pr(X_i \leq \theta) = 0.5\) for all i from 1 to n.
\end{definition}
\begin{definition}
S is defined as\\
\centering
\(S = \sum_{i=1}^{n} I(X_i > \theta_0)\)\\
where \(I(X_i > \theta_0\) represents an indicator function.
\end{definition}
\begin{align}
 I(X_i > \theta_0) &=
 \begin{cases}
 1, & \text{if } X_i > \theta_0 \\
 0, & \text{if } X_i \leq \theta_0
 \end{cases}
\end{align}
\begin{align}
E(S) &= E\left(\sum_{i=1}^{n} I(X_i > \theta_0)\right)\\
     &=\sum_{i=1}^{n} E(I(X_i > \theta_0))
\end{align}
Since,
\begin{align}
 E(I(X_i > \theta_0)) &= P(X_i > \theta_0) = \int_{\theta_0}^{\infty} f(x) \, dx
\end{align}
Therefore,
\begin{align}
E(S) &= \sum_{i=1}^{n} \int_{\theta_0}^{\infty} f(x) \, dx
\end{align}
\begin{enumerate}
\item From (6.12), under $H_0$, the distribution of $S$ depends on $f(\cdot)$.\\
\item The power function can be expressed as:
\begin{align}
\pi(\theta) &= \Pr(\text{Reject } H_0 \, | \, H_1 \text{ is true})\\
&= \Pr(S > c | \theta)
\end{align}
Therefore, power function depends on value of $\theta$.
\end{enumerate}


\item Let $X_1$, $X_2$, $X_3$....,$X_n$ be a random sample of size $n \brak{\geq 2}$ from a population having probability density function
\begin{align*}
f\brak{x;\theta}= \begin{cases} 
      \frac{2}{\theta x}\brak{\log_{e}x}e^{-\frac{\brak{\log_{e}x}^2}{\theta}} &, 0<x<1  \\
      0 &, otherwise
   \end{cases}
\end{align*}
where $\theta > 0$ is an unknown parameter. Then which of the following statements is true,\\
\begin{enumerate}[label=(\Alph*)]
\item $\frac{1}{n}\sum_{i=1}^{n}\brak{\ln X_i}^2$ is the maximum likelihood estimator of $\theta$
\item $\frac{1}{n-1}\sum_{i=1}^{n}\brak{\ln X_i}^2$ is the maximum likelihood estimator of $\theta$
\item $\frac{1}{n}\sum_{i=1}^{n}\ln X_i$ is the maximum likelihood estimator of $\theta$
\item $\frac{1}{n-1}\sum_{i=1}^{n}\ln X_i$ is the maximum likelihood estimator of $\theta$
\end{enumerate}
\hfill (GATE ST 2023)
\\

\let\negmedspace\undefined
\let\negthickspace\undefined
\documentclass[journal,12pt,twocolumn]{IEEEtran}
\usepackage{cite}
\usepackage{amsmath,amssymb,amsfonts,amsthm}
\usepackage{algorithmic}
\usepackage{graphicx}
\usepackage{textcomp}
\usepackage{xcolor}
\usepackage{txfonts}
\usepackage{listings}
\usepackage{enumitem}
\usepackage{mathtools}
\usepackage{gensymb}
\usepackage[breaklinks=true]{hyperref}
\usepackage{tkz-euclide} % loads  TikZ and tkz-base
\usepackage{listings}
\usepackage{float}

%
%\usepackage{setspace}
%\usepackage{gensymb}
%\doublespacing
%\singlespacing

\usepackage{graphicx}
%\usepackage{amssymb}
%\usepackage{relsize}
%\usepackage[cmex10]{amsmath}
%\usepackage{amsthm}
%\interdisplaylinepenalty=2500
%\savesymbol{iint}
%\usepackage{txfonts}
%\restoresymbol{TXF}{iint}
%\usepackage{wasysym}
%\usepackage{amsthm}
%\usepackage{iithtlc}
%\usepackage{mathrsfs}
%\usepackage{txfonts}
%\usepackage{stfloats}
%\usepackage{bm}
%\usepackage{cite}
%\usepackage{cases}
%\usepackage{subfig}
%\usepackage{xtab}
%\usepackage{longtable}
%\usepackage{multirow}
%\usepackage{algorithm}
%\usepackage{algpseudocode}
%\usepackage{enumitem}
%\usepackage{mathtools}
%\usepackage{tikz}
%\usepackage{circuitikz}
%\usepackage{verbatim}
%\usepackage{tfrupee}
%\usepackage{stmaryrd}
%\usetkzobj{all}
%    \usepackage{color}                                            %%
%    \usepackage{array}                                            %%
%    \usepackage{longtable}                                        %%
%    \usepackage{calc}                                             %%
%    \usepackage{multirow}                                         %%
%    \usepackage{hhline}                                           %%
%    \usepackage{ifthen}                                           %%
  %optionally (for landscape tables embedded in another document): %%
%    \usepackage{lscape}     
%\usepackage{multicol}
%\usepackage{chngcntr}
%\usepackage{enumerate}

%\usepackage{wasysym}
%\documentclass[conference]{IEEEtran}
%\IEEEoverridecommandlockouts
% The preceding line is only needed to identify funding in the first footnote. If that is unneeded, please comment it out.

\newtheorem{theorem}{Theorem}[section]
\newtheorem{problem}{Problem}
\newtheorem{proposition}{Proposition}[section]
\newtheorem{lemma}{Lemma}[section]
\newtheorem{corollary}[theorem]{Corollary}
\newtheorem{example}{Example}[section]
\newtheorem{definition}[problem]{Definition}
%\newtheorem{thm}{Theorem}[section] 
%\newtheorem{defn}[thm]{Definition}
%\newtheorem{algorithm}{Algorithm}[section]
%\newtheorem{cor}{Corollary}
\newcommand{\BEQA}{\begin{eqnarray}}
\newcommand{\EEQA}{\end{eqnarray}}
\newcommand{\define}{\stackrel{\triangle}{=}}
\theoremstyle{remark}
\newtheorem{rem}{Remark}
\parindent 0px

%\bibliographystyle{ieeetr}
\begin{document}
%
\providecommand{\pr}[1]{\ensuremath{\Pr\left(#1\right)}}
\providecommand{\prt}[2]{\ensuremath{p_{#1}^{\left(#2\right)} }}        % own macro for this question
\providecommand{\qfunc}[1]{\ensuremath{Q\left(#1\right)}}
\providecommand{\sbrak}[1]{\ensuremath{{}\left[#1\right]}}
\providecommand{\lsbrak}[1]{\ensuremath{{}\left[#1\right.}}
\providecommand{\rsbrak}[1]{\ensuremath{{}\left.#1\right]}}
\providecommand{\brak}[1]{\ensuremath{\left(#1\right)}}
\providecommand{\lbrak}[1]{\ensuremath{\left(#1\right.}}
\providecommand{\rbrak}[1]{\ensuremath{\left.#1\right)}}
\providecommand{\cbrak}[1]{\ensuremath{\left\{#1\right\}}}
\providecommand{\lcbrak}[1]{\ensuremath{\left\{#1\right.}}
\providecommand{\rcbrak}[1]{\ensuremath{\left.#1\right\}}}
\newcommand{\sgn}{\mathop{\mathrm{sgn}}}
\providecommand{\abs}[1]{\left\vert#1\right\vert}
\providecommand{\res}[1]{\Res\displaylimits_{#1}} 
\providecommand{\norm}[1]{\left\lVert#1\right\rVert}
%\providecommand{\norm}[1]{\lVert#1\rVert}
\providecommand{\mtx}[1]{\mathbf{#1}}
\providecommand{\mean}[1]{E\left[ #1 \right]}
\providecommand{\cond}[2]{#1\middle|#2}
\providecommand{\fourier}{\overset{\mathcal{F}}{ \rightleftharpoons}}
\newenvironment{amatrix}[1]{%
  \left(\begin{array}{@{}*{#1}{c}|c@{}}
}{%
  \end{array}\right)
}
%\providecommand{\hilbert}{\overset{\mathcal{H}}{ \rightleftharpoons}}
%\providecommand{\system}{\overset{\mathcal{H}}{ \longleftrightarrow}}
	%\newcommand{\solution}[2]{\textbf{Solution:}{#1}}
\newcommand{\solution}{\noindent \textbf{Solution: }}
\newcommand{\cosec}{\,\text{cosec}\,}
\providecommand{\dec}[2]{\ensuremath{\overset{#1}{\underset{#2}{\gtrless}}}}
\newcommand{\myvec}[1]{\ensuremath{\begin{pmatrix}#1\end{pmatrix}}}
\newcommand{\mydet}[1]{\ensuremath{\begin{vmatrix}#1\end{vmatrix}}}
\newcommand{\myaugvec}[2]{\ensuremath{\begin{amatrix}{#1}#2\end{amatrix}}}
\providecommand{\rank}{\text{rank}}
\providecommand{\pr}[1]{\ensuremath{\Pr\left(#1\right)}}
\providecommand{\qfunc}[1]{\ensuremath{Q\left(#1\right)}}
	\newcommand*{\permcomb}[4][0mu]{{{}^{#3}\mkern#1#2_{#4}}}
\newcommand*{\perm}[1][-3mu]{\permcomb[#1]{P}}
\newcommand*{\comb}[1][-1mu]{\permcomb[#1]{C}}
\providecommand{\qfunc}[1]{\ensuremath{Q\left(#1\right)}}
\providecommand{\gauss}[2]{\mathcal{N}\ensuremath{\left(#1,#2\right)}}
\providecommand{\diff}[2]{\ensuremath{\frac{d{#1}}{d{#2}}}}
\providecommand{\myceil}[1]{\left \lceil #1 \right \rceil }
\newcommand\figref{Fig.~\ref}
\newcommand\tabref{Table~\ref}
\newcommand{\sinc}{\,\text{sinc}\,}
\newcommand{\rect}{\,\text{rect}\,}
%%
%	%\newcommand{\solution}[2]{\textbf{Solution:}{#1}}
%\newcommand{\solution}{\noindent \textbf{Solution: }}
%\newcommand{\cosec}{\,\text{cosec}\,}
%\numberwithin{equation}{section}
%\numberwithin{equation}{subsection}
%\numberwithin{problem}{section}
%\numberwithin{definition}{section}
%\makeatletter
%\@addtoreset{figure}{problem}
%\makeatother

%\let\StandardTheFigure\thefigure
\let\vec\mathbf


\bibliographystyle{IEEEtran}


\vspace{3cm}

\title{
%	\logo{
EE23010 NCERT Exemplar
%	}
}
\author{Vishal A - EE22BTECH11057}

	
	

%\title{
%	\logo{Matrix Analysis through Octave}{\begin{center}\includegraphics[scale=.24]{tlc}\end{center}}{}{HAMDSP}
%}


% paper title
% can use linebreaks \\ within to get better formatting as desired
%\title{Matrix Analysis through Octave}
%
%
% author names and IEEE memberships
% note positions of commas and nonbreaking spaces ( ~ ) LaTeX will not break
% a structure at a ~ so this keeps an author's name from being broken across
% two lines.
% use \thanks{} to gain access to the first footnote area
% a separate \thanks must be used for each paragraph as LaTeX2e's \thanks
% was not built to handle multiple paragraphs
%

%\author{<-this % stops a space
%\thanks{}}
%}
% note the % following the last \IEEEmembership and also \thanks - 
% these prevent an unwanted space from occurring between the last author name
% and the end of the author line. i.e., if you had this:
% 
% \author{....lastname \thanks{...} \thanks{...} }
%                     ^------------^------------^----Do not want these spaces!
%
% a space would be appended to the last name and could cause every name on that
% line to be shifted left slightly. This is one of those "LaTeX things". For
% instance, "\textbf{A} \textbf{B}" will typeset as "A B" not "AB". To get
% "AB" then you have to do: "\textbf{A}\textbf{B}"
% \thanks is no different in this regard, so shield the last } of each \thanks
% that ends a line with a % and do not let a space in before the next \thanks.
% Spaces after \IEEEmembership other than the last one are OK (and needed) as
% you are supposed to have spaces between the names. For what it is worth,
% this is a minor point as most people would not even notice if the said evil
% space somehow managed to creep in.



% The paper headers
%\markboth{Journal of \LaTeX\ Class Files,~Vol.~6, No.~1, January~2007}%
%{Shell \MakeLowercase{\textit{et al.}}: Bare Demo of IEEEtran.cls for Journals}
% The only time the second header will appear is for the odd numbered pages
% after the title page when using the twoside option.
% 
% *** Note that you probably will NOT want to include the author's ***
% *** name in the headers of peer review papers.                   ***
% You can use \ifCLASSOPTIONpeerreview for conditional compilation here if
% you desire.




% If you want to put a publisher's ID mark on the page you can do it like
% this:
%\IEEEpubid{0000--0000/00\$00.00~\copyright~2007 IEEE}
% Remember, if you use this you must call \IEEEpubidadjcol in the second
% column for its text to clear the IEEEpubid mark.



% make the title area
\maketitle
\textbf{Question 23.2023}\\
Let $X_1$, $X_2$, $X_3$....,$X_n$ be a random sample of size $n \brak{\geq 2}$ from a population having probability density function
\begin{align*}
f\brak{x;\theta}= \begin{cases} 
      \frac{2}{\theta x}\brak{\log_{e}x}e^{-\frac{\brak{\log_{e}x}^2}{\theta}} &, 0<x<1  \\
      0 &, otherwise
   \end{cases}
\end{align*}
where $\theta > 0$ is an unknown parameter. Then which of the following statements is true,\\
\begin{enumerate}[label=(\Alph*)]
\item $\frac{1}{n}\sum_{i=1}^{n}\brak{\ln X_i}^2$ is the maximum likelihood estimator of $\theta$
\item $\frac{1}{n-1}\sum_{i=1}^{n}\brak{\ln X_i}^2$ is the maximum likelihood estimator of $\theta$
\item $\frac{1}{n}\sum_{i=1}^{n}\ln X_i$ is the maximum likelihood estimator of $\theta$
\item $\frac{1}{n-1}\sum_{i=1}^{n}\ln X_i$ is the maximum likelihood estimator of $\theta$
\end{enumerate}
\solution
\begin{align}
L\brak{\theta} &= f\brak{x_1,x_2,..,x_n;\theta}
\end{align}
The product of pdfs can be used to approximate the likelihood function even if the variables are dependent. This is a general approach that is often used in practice to estimate MLE of $\theta$. Therefore,
\begin{align}
L\brak{\theta} = \prod_{i=1}^n f\brak{x_i;\theta}
\end{align}
Maximizing $L\brak{\theta}$ is equivalent to maximizing the the $\ln L\brak{\theta}$ as $\ln$ is a monotonically increasing function.
\begin{align}
l\brak{\theta} &= \ln L\brak{\theta}\\
&= \ln\brak{\prod_{i=1}^n f\brak{x_i;\theta}}\\
&= \sum_{i=1}^{n} \ln f\brak{x_i;\theta}\\
&= - n \ln 2 - n\ln \theta + \sum_{i=1}^{n}\ln \brak{-\ln x_{i}} - \sum_{i=1}^{n} \brak{\ln x_i} -\sum_{i=1}^{n}\frac{(\ln x_i)^2}{\theta} 
\end{align}
Maximizing $l\brak{\theta}$ with respect to $\theta$ gives the MLE estimation, therefore
\begin{align}
\frac{\partial l\brak{\theta}}{\partial \theta} &= 0\\
\frac{-n}{\theta} + \frac{1}{\brak{\theta}^2}\sum_{i=1}^{n}\brak{\ln X_i}^2 &= 0\\
\theta &= \frac{1}{n}\sum_{i=1}^{n}\brak{\ln X_i}^2
\end{align}
Hence (A) is the true statement.













% For peer review papers, you can put extra information on the cover
% page as needed:
% \ifCLASSOPTIONpeerreview
% \begin{center} \bfseries EDICS Category: 3-BBND \end{center}
% \fi
%
% For peerreview papers, this IEEEtran command inserts a page break and
% creates the second title. It will be ignored for other modes.
%\IEEEpeerreviewmaketitle

\end{document}
\item Suppose that $(X, Y)$ has joint probability mass function
\begin{align}
P(X = 0, Y = 0) &= P(X = 1, Y = 1) = \theta, \\
P(X = 1, Y = 0) &= P(X = 0, Y = 1) = \frac{1}{2} - \theta.
\end{align}
where $0 \le \theta \le \frac{1}{2}$ is an unknown parameter. Consider testing $H_0 : \theta = \frac{1}{4}$ against $H_1 : \theta = \frac{1}{3}$; based on a random sample ${(X_1 , Y_1 ), (X_2 , Y_2 ), \ldots (X_n , Y_n )}$ from the above probability mass function. Let $M$ be the cardinality of the set $\{i: X_i = Y_i , 1 \le i\le n\}$. If $m$ is the observed value of $M$, then which one of the following statements is true?
\begin{enumerate}
\item The likelihood ratio test rejects $H_0$ if $m > c$ for some $c$.
\item The likelihood ratio test rejects $H_0$ if $m < c$ for some $c$.
\item The likelihood ratio test rejects $H_0$ if $c_1 < m < c_2$ for some $c_1$ and $c_2$.
\item The likelihood ratio test rejects $H_0$ if $m < c_1$ or $m > c_2$ for some $c_1$ and $c_2$.
\end{enumerate}
\hfill (GATE ST 2023)
\\
\iffalse
\let\negmedspace\undefined
\let\negthickspace\undefined
\documentclass[article]{IEEEtran}
       \def\inputGnumericTable{}                                 %%
\usepackage{cite}
\usepackage{amsmath,amssymb,amsfonts,amsthm}
\usepackage{algorithmic}
\usepackage{graphicx}
\usepackage{textcomp}
\usepackage{xcolor}
\usepackage{txfonts}
\usepackage{listings}
\usepackage{enumitem}
\usepackage{mathtools}
\usepackage{gensymb}
\usepackage[breaklinks=true]{hyperref}
\usepackage{tkz-euclide} % loads  TikZ and tkz-base
\usepackage{listings}
\renewcommand{\theenumi}{\Alph{enumi}}
%
%\usepackage{setspace}
%\usepackage{gensymb}
%\doublespacing
%\singlespacing

%\usepackage{graphicx}
%\usepackage{amssymb}
%\usepackage{relsize}
%\usepackage[cmex10]{amsmath}
%\usepackage{amsthm}
%\interdisplaylinepenalty=2500
%\savesymbol{iint}
%\usepackage{txfonts}
%\restoresymbol{TXF}{iint}
%\usepackage{wasysym}
%\usepackage{amsthm}
%\usepackage{iithtlc}
%\usepackage{mathrsfs}
%\usepackage{txfonts}
%\usepackage{stfloats}
%\usepackage{bm}
%\usepackage{cite}
%\usepackage{cases}
%\usepackage{subfig}
%\usepackage{xtab}
%\usepackage{longtable}
%\usepackage{multirow}
%\usepackage{algorithm}
%\usepackage{algpseudocode}
%\usepackage{enumitem}
%\usepackage{mathtools}
%\usepackage{tikz}
%\usepackage{circuitikz}
%\usepackage{verbatim}
%\usepackage{tfrupee}
%\usepackage{stmaryrd}
%\usetkzobj{all}
    \usepackage{color}                                            %%
    \usepackage{array}                                            %%
    \usepackage{longtable}                                        %%
    \usepackage{calc}                                             %%
    \usepackage{multirow}                                         %%
    \usepackage{hhline}                                           %%
    \usepackage{ifthen}                                           %%
 %optionally (for landscape tables embedded in another document): %%
    \usepackage{lscape}     
%\usepackage{multicol}
%\usepackage{chngcntr}
%\usepackage{enumerate}

%\usepackage{wasysym}
%\documentclass[conference]{IEEEtran}
%\IEEEoverridecommandlockouts
% The preceding line is only needed to identify funding in the first footnote. If that is unneeded, please comment it out.

\newtheorem{theorem}{Theorem}[section]
\newtheorem{problem}{Problem}
\newtheorem{proposition}{Proposition}[section]
\newtheorem{lemma}{Lemma}[section]
\newtheorem{corollary}[theorem]{Corollary}
\newtheorem{example}{Example}[section]
\newtheorem{definition}[problem]{Definition}
%\newtheorem{thm}{Theorem}[section] 
%\newtheorem{defn}[thm]{Definition}
%\newtheorem{algorithm}{Algorithm}[section]
%\newtheorem{cor}{Corollary}
\newcommand{\BEQA}{\begin{eqnarray}}
\newcommand{\EEQA}{\end{eqnarray}}
\newcommand{\define}{\stackrel{\triangle}{=}}
\theoremstyle{remark}
\newtheorem{rem}{Remark}

\begin{document}
\providecommand{\pr}[1]{\ensuremath{\Pr\left(#1\right)}}
\providecommand{\prt}[2]{\ensuremath{p_{#1}^{\left(#2\right)} }}        % own macro for this question
\providecommand{\qfunc}[1]{\ensuremath{Q\left(#1\right)}}
\providecommand{\sbrak}[1]{\ensuremath{{}\left[#1\right]}}
\providecommand{\lsbrak}[1]{\ensuremath{{}\left[#1\right.}}
\providecommand{\rsbrak}[1]{\ensuremath{{}\left.#1\right]}}
\providecommand{\brak}[1]{\ensuremath{\left(#1\right)}}
\providecommand{\lbrak}[1]{\ensuremath{\left(#1\right.}}
\providecommand{\rbrak}[1]{\ensuremath{\left.#1\right)}}
\providecommand{\cbrak}[1]{\ensuremath{\left\{#1\right\}}}
\providecommand{\lcbrak}[1]{\ensuremath{\left\{#1\right.}}
\providecommand{\rcbrak}[1]{\ensuremath{\left.#1\right\}}}
\newcommand{\sgn}{\mathop{\mathrm{sgn}}}
\providecommand{\abs}[1]{\left\vert#1\right\vert}
\providecommand{\res}[1]{\Res\displaylimits_{#1}} 
\providecommand{\norm}[1]{\left\lVert#1\right\rVert}
%\providecommand{\norm}[1]{\lVert#1\rVert}
\providecommand{\mtx}[1]{\mathbf{#1}}
\providecommand{\mean}[1]{E\left[ #1 \right]}
\providecommand{\cond}[2]{#1\middle|#2}
\providecommand{\fourier}{\overset{\mathcal{F}}{ \rightleftharpoons}}
\newenvironment{amatrix}[1]{%
  \left(\begin{array}{@{}*{#1}{c}|c@{}}
}{%
  \end{array}\right)
}
%\providecommand{\hilbert}{\overset{\mathcal{H}}{ \rightleftharpoons}}
%\providecommand{\system}{\overset{\mathcal{H}}{ \longleftrightarrow}}
	%\newcommand{\solution}[2]{\textbf{Solution:}{#1}}
\newcommand{\solution}{\noindent \textbf{Solution: }}
\newcommand{\cosec}{\,\text{cosec}\,}
\providecommand{\dec}[2]{\ensuremath{\overset{#1}{\underset{#2}{\gtrless}}}}
\newcommand{\myvec}[1]{\ensuremath{\begin{pmatrix}#1\end{pmatrix}}}
\newcommand{\mydet}[1]{\ensuremath{\begin{vmatrix}#1\end{vmatrix}}}
\newcommand{\myaugvec}[2]{\ensuremath{\begin{amatrix}{#1}#2\end{amatrix}}}
\providecommand{\rank}{\text{rank}}
\providecommand{\pr}[1]{\ensuremath{\Pr\left(#1\right)}}
\providecommand{\qfunc}[1]{\ensuremath{Q\left(#1\right)}}
	\newcommand*{\permcomb}[4][0mu]{{{}^{#3}\mkern#1#2_{#4}}}
\newcommand*{\perm}[1][-3mu]{\permcomb[#1]{P}}
\newcommand*{\comb}[1][-1mu]{\permcomb[#1]{C}}
\providecommand{\qfunc}[1]{\ensuremath{Q\left(#1\right)}}
\providecommand{\gauss}[2]{\mathcal{N}\ensuremath{\left(#1,#2\right)}}
\providecommand{\diff}[2]{\ensuremath{\frac{d{#1}}{d{#2}}}}
\providecommand{\myceil}[1]{\left \lceil #1 \right \rceil }
\newcommand\figref{Fig.~\ref}
\newcommand\tabref{Table~\ref}
\newcommand{\sinc}{\,\text{sinc}\,}
\newcommand{\rect}{\,\text{rect}\,}
%%
%	%\newcommand{\solution}[2]{\textbf{Solution:}{#1}}
%\newcommand{\solution}{\noindent \textbf{Solution: }}
%\newcommand{\cosec}{\,\text{cosec}\,}
%\numberwithin{equation}{section}
%\numberwithin{equation}{subsection}
%\numberwithin{problem}{section}
%\numberwithin{definition}{section}
%\makeatletter
%\@addtoreset{figure}{problem}
%\makeatother

%\let\StandardTheFigure\thefigure
\let\vec\mathbf

\bibliographystyle{IEEEtran}
\title{
%	\logo{
Assignment
%	}
}
\author{ Karthikeya hanu prakash kanithi (EE22BTECH11026)}
\maketitle
\parindent0px
\vspace{3cm}
Question : Suppose that $(X, Y)$ has joint probability mass function
\begin{align}
P(X = 0, Y = 0) &= P(X = 1, Y = 1) = \theta, \\
P(X = 1, Y = 0) &= P(X = 0, Y = 1) = \frac{1}{2} - \theta.
\end{align}
where $0 \le \theta \le \frac{1}{2}$ is an unknown parameter. Consider testing $H_0 : \theta = \frac{1}{4}$ against $H_1 : \theta = \frac{1}{3}$; based on a random sample ${(X_1 , Y_1 ), (X_2 , Y_2 ), \ldots (X_n , Y_n )}$ from the above probability mass function. Let $M$ be the cardinality of the set $\{i: X_i = Y_i , 1 \le i\le n\}$. If $m$ is the observed value of $M$, then which one of the following statements is true?
\begin{enumerate}
\item The likelihood ratio test rejects $H_0$ if $m > c$ for some $c$.
\item The likelihood ratio test rejects $H_0$ if $m < c$ for some $c$.
\item The likelihood ratio test rejects $H_0$ if $c_1 < m < c_2$ for some $c_1$ and $c_2$.
\item The likelihood ratio test rejects $H_0$ if $m < c_1$ or $m > c_2$ for some $c_1$ and $c_2$.
\end{enumerate}
\fi
\solution 
Given that,
\begin{align}
	H_0 : \quad \theta = \theta_0 = \frac{1}{4},\\
	H_1 : \quad \theta = \theta_1 = \frac{1}{3}.
\end{align}
and the pmf is given by 
\begin{align}
	p_{XY}(0,0) &= p_{XY}(1,1) = \theta \\
	p_{XY}(0,1) &= p_{XY}(1,0) = \frac{1}{2} - \theta 
\end{align}
Then for the given random sample of data, 
\begin{align}
    p_{X_i,Y_i}(x,y) &= 
    \begin{cases}
        2\theta &  x=y  \\
        1 - 2\theta & x\ne y
    \end{cases} \\
\end{align}
Then the likelihood of the data under $H_0$ is given by: 
\begin{align}
    L(\theta_0 \mid data) &= \prod_{i=1}^{n} p_{X_i,Y_i}(x,y) \\
    &= \brak{2\theta_0}^m\brak{1 - 2\theta_0}^{n-m}\\
    &= \brak{\frac{1}{2}}^m\brak{\frac{1}{2}}^{n-m}
\end{align}
Then the likelihood of the data under $H_1$ is given by:
\begin{align}
    L(\theta_1 \mid data) &= \prod_{i=1}^{n} p_{X_i,Y_i}(x,y) \\
    &= \brak{2\theta_1}^m\brak{1 - 2\theta_1}^{n-m}\\
    &= \brak{\frac{2}{3}}^m\brak{\frac{1}{3}}^{n-m}
\end{align}
The likelyhood ratio will be 
\begin{align}
    \lambda(data) &= \frac{L(\theta_1 \mid x)}{L(\theta_0 \mid x)} \\
    &= \frac{\brak{\frac{2}{3}}^m\brak{\frac{1}{3}}^{n-m}}{\brak{\frac{1}{2}}^m\brak{\frac{1}{2}}^{n-m}} = \brak{2}^m\brak{\frac{2}{3}}^{n} \label{eq:st/42/1}
\end{align}
Let the critical value be denoted by $c_1$, then the likelihood ratio test rejects $H_0$ if
\begin{align}
    \implies  \lambda(data) &\overset{H_1}{\underset{H_0}{\gtrless}} c_1\\
\end{align}  
From \eqref{eq:st/42/1},
\begin{align}
    \implies  \brak{2}^m\brak{\frac{2}{3}}^{n} &\overset{H_1}{\underset{H_0}{\gtrless}} c_1\\
    \implies  \brak{2}^m &\overset{H_1}{\underset{H_0}{\gtrless}} c_1\brak{\frac{2}{3}}^{n}\\
    \implies  m &\overset{H_1}{\underset{H_0}{\gtrless}} \log_{2}\brak{c_1\brak{\frac{2}{3}}}^{n}\\
    \implies  m &\overset{H_1}{\underset{H_0}{\gtrless}} c \quad \exists \, c \in \mathbb{R} \label{eq:st/42/2}
\end{align}
where, 
\begin{align}
    c = \log_{2}\brak{c_1\brak{\frac{2}{3}}}^{n}
\end{align}
$\therefore$ From \eqref{eq:st/42/2}, Option A is correct and Options B,C,D are incorrect


\item Let $X$ be a random sample of size 1 from a population with cumulative distribution function
\begin{align}
F_X\brak{x} = 
\begin{cases}
0 & \text{if } x \leq 0\\
1-\brak{1-x}^{\theta} & \text{if } 0 \leq x < 1\\
1 & \text{if } x \geq 1,
\end{cases}
\end{align}
where $\theta > 0$ is an unknown parameter. To test $H_0:\theta = 1$ against $H_1:\theta = 2$, consider using the critical region $\brak{x \in \mathbb{R}: x < 0.5}$. If $\alpha$ and $\beta$ denote the level and power of the test, respectively, then $\alpha + \beta$ (rounded off to two decimal places) equals
\hfill(GATE ST 2023)
\\
\let\negmedspace\undefined
\let\negthickspace\undefined
\def\inputGnumericTable{}  
\documentclass[journal,12pt,twocolumn]{IEEEtran}
\usepackage{cite}
\usepackage{amsmath,amssymb,amsfonts,amsthm}
\usepackage{algorithmic}
\usepackage{graphicx}
\usepackage{textcomp}
\usepackage{xcolor}
\usepackage{txfonts}
\usepackage{listings}
\usepackage{enumitem}
\usepackage{mathtools}
\usepackage{gensymb}
\usepackage[breaklinks=true]{hyperref}
\usepackage{tkz-euclide} % loads  TikZ and tkz-base
\usepackage{listings}
\usepackage{gvv}
\usepackage[latin1]{inputenc}                                 
\usepackage{color}                                            
\usepackage{array}                                            
\usepackage{longtable}                                        
\usepackage{calc}                                             
\usepackage{multirow}                                         
\usepackage{hhline}                                           
\usepackage{ifthen}                                           
\usepackage{lscape}  
%
%\usepackage{setspace}
%\usepackage{gensymb}
%\doublespacing
%\singlespacing

%\usepackage{graphicx}
%\usepackage{amssymb}
%\usepackage{relsize}
%\usepackage[cmex10]{amsmath}
%\usepackage{amsthm}
%\interdisplaylinepenalty=2500
%\savesymbol{iint}
%\usepackage{txfonts}
%\restoresymbol{TXF}{iint}
%\usepackage{wasysym}
%\usepackage{amsthm}
%\usepackage{iithtlc}
%\usepackage{mathrsfs}
%\usepackage{txfonts}
%\usepackage{stfloats}
%\usepackage{bm}
%\usepackage{cite}
%\usepackage{cases}
%\usepackage{subfig}
%\usepackage{xtab}
%\usepackage{longtable}
%\usepackage{multirow}
%\usepackage{algorithm}
%\usepackage{algpseudocode}
%\usepackage{enumitem}
%\usepackage{mathtools}
%\usepackage{tikz}
%\usepackage{circuitikz}
%\usepackage{verbatim}
%\usepackage{tfrupee}
%\usepackage{stmaryrd}
%\usetkzobj{all}
%    \usepackage{color}                                            %%
%    \usepackage{array}                                            %%
%    \usepackage{longtable}                                        %%
%    \usepackage{calc}                                             %%
%    \usepackage{multirow}                                         %%
%    \usepackage{hhline}                                           %%
%    \usepackage{ifthen}                                           %%
  %optionally (for landscape tables embedded in another document): %%
%    \usepackage{lscape}     
%\usepackage{multicol}
%\usepackage{chngcntr}
%\usepackage{enumerate}

%\usepackage{wasysym}
%\documentclass[conference]{IEEEtran}
%\IEEEoverridecommandlockouts
% The preceding line is only needed to identify funding in the first footnote. If that is unneeded, please comment it out.

\newtheorem{theorem}{Theorem}[section]
\newtheorem{problem}{Problem}
\newtheorem{proposition}{Proposition}[section]
\newtheorem{lemma}{Lemma}[section]
\newtheorem{corollary}[theorem]{Corollary}
\newtheorem{example}{Example}[section]
\newtheorem{definition}[problem]{Definition}
%\newtheorem{thm}{Theorem}[section] 
%\newtheorem{defn}[thm]{Definition}
%\newtheorem{algorithm}{Algorithm}[section]
%\newtheorem{cor}{Corollary}
\newcommand{\BEQA}{\begin{eqnarray}}
\newcommand{\EEQA}{\end{eqnarray}}
\newcommand{\define}{\stackrel{\triangle}{=}}
\theoremstyle{remark}
\newtheorem{rem}{Remark}

%\bibliographystyle{ieeetr}
\begin{document}
%

\bibliographystyle{IEEEtran}


\vspace{3cm}

\title{
%	\logo{
Answer 
%	}
}
\author{ Dhruv Parashar - EE22BTECH11019$^{*}$% <-this % stops a space	
}	
%\title{
%	\logo{Matrix Analysis through Octave}{\begin{center}\includegraphics[scale=.24]{tlc}\end{center}}{}{HAMDSP}
%}


% paper title
% can use linebreaks \\ within to get better formatting as desired
%\title{Matrix Analysis through Octave}
%
%
% author names and IEEE memberships
% note positions of commas and nonbreaking spaces ( ~ ) LaTeX will not break
% a structure at a ~ so this keeps an author's name from being broken across
% two lines.
% use \thanks{} to gain access to the first footnote area
% a separate \thanks must be used for each paragraph as LaTeX2e's \thanks
% was not built to handle multiple paragraphs
%

%\author{<-this % stops a space
%\thanks{}}
%}
% note the % following the last \IEEEmembership and also \thanks - 
% these prevent an unwanted space from occurring between the last author name
% and the end of the author line. i.e., if you had this:
% 
% \author{....lastname \thanks{...} \thanks{...} }
%                     ^------------^------------^----Do not want these spaces!
%
% a space would be appended to the last name and could cause every name on that
% line to be shifted left slightly. This is one of those "LaTeX things". For
% instance, "\textbf{A} \textbf{B}" will typeset as "A B" not "AB". To get
% "AB" then you have to do: "\textbf{A}\textbf{B}"
% \thanks is no different in this regard, so shield the last } of each \thanks
% that ends a line with a % and do not let a space in before the next \thanks.
% Spaces after \IEEEmembership other than the last one are OK (and needed) as
% you are supposed to have spaces between the names. For what it is worth,
% this is a minor point as most people would not even notice if the said evil
% space somehow managed to creep in.



% The paper headers
%\markboth{Journal of \LaTeX\ Class Files,~Vol.~6, No.~1, January~2007}%
%{Shell \MakeLowercase{\textit{et al.}}: Bare Demo of IEEEtran.cls for Journals}
% The only time the second header will appear is for the odd numbered pages
% after the title page when using the twoside option.
% 
% *** Note that you probably will NOT want to include the author's ***
% *** name in the headers of peer review papers.                   ***
% You can use \ifCLASSOPTIONpeerreview for conditional compilation here if
% you desire.




% If you want to put a publisher's ID mark on the page you can do it like
% this:
%\IEEEpubid{0000--0000/00\$00.00~\copyright~2007 IEEE}
% Remember, if you use this you must call \IEEEpubidadjcol in the second
% column for its text to clear the IEEEpubid mark.



% make the title area
\maketitle



%\tableofcontents

\bigskip

\renewcommand{\thefigure}{\theenumi}
\renewcommand{\thetable}{\theenumi}
%\renewcommand{\theequation}{\theenumi}

%\begin{abstract}
%%\boldmath
%In this letter, an algorithm for evaluating the exact analytical bit error rate  (BER)  for the piecewise linear (PL) combiner for  multiple relays is presented. Previous results were available only for upto three relays. The algorithm is unique in the sense that  the actual mathematical expressions, that are prohibitively large, need not be explicitly obtained. The diversity gain due to multiple relays is shown through plots of the analytical BER, well supported by simulations. 
%
%\end{abstract}
% IEEEtran.cls defaults to using nonbold math in the Abstract.
% This preserves the distinction between vectors and scalars. However,
% if the journal you are submitting to favors bold math in the abstract,
% then you can use LaTeX's standard command \boldmath at the very start
% of the abstract to achieve this. Many IEEE journals frown on math
% in the abstract anyway.

% Note that keywords are not normally used for peerreview papers.
%\begin{IEEEkeywords}
%Cooperative diversity, decode and forward, piecewise linear
%\end{IEEEkeywords}



% For peer review papers, you can put extra information on the cover
% page as needed:
% \ifCLASSOPTIONpeerreview
% \begin{center} \bfseries EDICS Category: 3-BBND \end{center}
% \fi
%
% For peerreview papers, this IEEEtran command inserts a page break and
% creates the second title. It will be ignored for other modes.
%\IEEEpeerreviewmaketitle
Question:
Let $X$ be a random sample of size 1 from a population with cumulative distribution function
\begin{align}
F_X\brak{x} = 
\begin{cases}
0 & \text{if } x \leq 0\\
1-\brak{1-x}^{\theta} & \text{if } 0 \leq x < 1\\
1 & \text{if } x \geq 1,
\end{cases}
\end{align}
where $\theta > 0$ is an unknown parameter. To test $H_0:\theta = 1$ against $H_1:\theta = 2$, consider using the critical region $\brak{x \in \mathbb{R}: x < 0.5}$. If $\alpha$ and $\beta$ denote the level and power of the test, respectively, then $\alpha + \beta$ (rounded off to two decimal places) equals
\solution
Given that,
\begin{align}
H_0: \theta &= \theta_0 = 1\\
H_1: \theta &= \theta_1 = 2
\end{align}
PDF can be defined as:
\begin{align}
p_X\brak{x} &= \frac{d}{dx}F_X\brak{x}\\
&= 
\begin{cases}
\theta\brak{1-x}^{\theta-1} & \text{if } 0 \leq x <1\\
0 & \text{otherwise}
\end{cases}
\end{align}
Level of test:
\begin{align}
\alpha &= \pr{\left.{\text{reject }H_0}|\right.{H_0\text{ is true}}}\\
&= \pr{\left.{x < 0.5}|\right.{\theta_0}}\\
&= F_X\brak{0.5}\\
&= 1 - \brak{1-0.5}\\
&= \frac{1}{2}
\end{align}
Power of test:
\begin{align}
\beta &= \pr{\left.{\text{reject }H_0}|\right.{H_1\text{ is true}}}\\
&= \pr{\left.{x < 0.5}| \right.{\theta_1}}\\
&= F_X\brak{0.5}\\
&= 1 - \brak{1-0.5}^2\\
&= \frac{3}{4}
\end{align}
Now,
\begin{align}
\alpha + \beta &= \frac{1}{2} + \frac{3}{4}\\
&= 1.25
\end{align}
\end{document}

\item Using the Ordinary Least Squares (OLS) method, a researcher estimated the relationship between initial salary (S) of MBA graduates and their cumulative grade point average (CGPA) as
\begin{align*}
\hat{S}_i = \hat{\beta}_0 + \hat{\beta}_1 \text{CGPA}_i ,  i = 1,2, \dots , 100
\end{align*}
where $\hat{\beta}_0 = 4543$ and $\hat{\beta}_1 = 645.08$. The standard errors of $\hat{\beta}_0$ and $\hat{\beta}_1$ are 921.79 and 70.01, respectively.\\
The t-statistic for testing the null hypothesis $\beta_1 = 0$ is
\hfill (GATE XH 2023)
\\
\iffalse
\let\negmedspace\undefined
\let\negthickspace\undefined
\documentclass[journal,12pt,twocolumn]{IEEEtran}
\usepackage{cite}
\usepackage{amsmath,amssymb,amsfonts,amsthm}
\usepackage{algorithmic}
\usepackage{graphicx}
\usepackage{textcomp}
\usepackage{xcolor}
\usepackage{txfonts}
\usepackage{listings}
\usepackage{enumitem}
\usepackage{mathtools}
\usepackage{gensymb}
\usepackage[breaklinks=true]{hyperref}
\usepackage{tkz-euclide} % loads  TikZ and tkz-base
\usepackage{listings}
\usepackage{float}

%
%\usepackage{setspace}
%\usepackage{gensymb}
%\doublespacing
%\singlespacing

\usepackage{graphicx}
%\usepackage{amssymb}
%\usepackage{relsize}
%\usepackage[cmex10]{amsmath}
%\usepackage{amsthm}
%\interdisplaylinepenalty=2500
%\savesymbol{iint}
%\usepackage{txfonts}
%\restoresymbol{TXF}{iint}
%\usepackage{wasysym}
\usepackage{amsthm}
%\usepackage{iithtlc}
%\usepackage{mathrsfs}
%\usepackage{txfonts}
%\usepackage{stfloats}
%\usepackage{bm}
%\usepackage{cite}
%\usepackage{cases}
%\usepackage{subfig}
%\usepackage{xtab}
%\usepackage{longtable}
%\usepackage{multirow}
%\usepackage{algorithm}
%\usepackage{algpseudocode}
%\usepackage{enumitem}
%\usepackage{mathtools}
%\usepackage{tikz}
%\usepackage{circuitikz}
%\usepackage{verbatim}
%\usepackage{tfrupee}
%\usepackage{stmaryrd}
%\usetkzobj{all}
%    \usepackage{color}                                            %%
%    \usepackage{array}                                            %%
%    \usepackage{longtable}                                        %%
%    \usepackage{calc}                                             %%
%    \usepackage{multirow}                                         %%
%    \usepackage{hhline}                                           %%
%    \usepackage{ifthen}                                           %%
  %optionally (for landscape tables embedded in another document): %%
%    \usepackage{lscape}     
%\usepackage{multicol}
%\usepackage{chngcntr}
%\usepackage{enumerate}

%\usepackage{wasysym}
%\documentclass[conference]{IEEEtran}
%\IEEEoverridecommandlockouts
% The preceding line is only needed to identify funding in the first footnote. If that is unneeded, please comment it out.

\newtheorem{theorem}{Theorem}[section]
\newtheorem{problem}{Problem}
\newtheorem{proposition}{Proposition}[section]
\newtheorem{lemma}{Lemma}[section]
\newtheorem{corollary}[theorem]{Corollary}
\newtheorem{example}{Example}[section]
\newtheorem{definition}[problem]{Definition}
%\newtheorem{thm}{Theorem}[section] 
%\newtheorem{defn}[thm]{Definition}
%\newtheorem{algorithm}{Algorithm}[section]
%\newtheorem{cor}{Corollary}
\newcommand{\BEQA}{\begin{eqnarray}}
\newcommand{\EEQA}{\end{eqnarray}}
\newcommand{\define}{\stackrel{\triangle}{=}}
\theoremstyle{remark}
\newtheorem{rem}{Remark}
\parindent 0px

%\bibliographystyle{ieeetr}
\begin{document}
%
\providecommand{\pr}[1]{\ensuremath{\Pr\left(#1\right)}}
\providecommand{\prt}[2]{\ensuremath{p_{#1}^{\left(#2\right)} }}        % own macro for this question
\providecommand{\qfunc}[1]{\ensuremath{Q\left(#1\right)}}
\providecommand{\sbrak}[1]{\ensuremath{{}\left[#1\right]}}
\providecommand{\lsbrak}[1]{\ensuremath{{}\left[#1\right.}}
\providecommand{\rsbrak}[1]{\ensuremath{{}\left.#1\right]}}
\providecommand{\brak}[1]{\ensuremath{\left(#1\right)}}
\providecommand{\lbrak}[1]{\ensuremath{\left(#1\right.}}
\providecommand{\rbrak}[1]{\ensuremath{\left.#1\right)}}
\providecommand{\cbrak}[1]{\ensuremath{\left\{#1\right\}}}
\providecommand{\lcbrak}[1]{\ensuremath{\left\{#1\right.}}
\providecommand{\rcbrak}[1]{\ensuremath{\left.#1\right\}}}
\newcommand{\sgn}{\mathop{\mathrm{sgn}}}
\providecommand{\abs}[1]{\left\vert#1\right\vert}
\providecommand{\res}[1]{\Res\displaylimits_{#1}} 
\providecommand{\norm}[1]{\left\lVert#1\right\rVert}
%\providecommand{\norm}[1]{\lVert#1\rVert}
\providecommand{\mtx}[1]{\mathbf{#1}}
\providecommand{\mean}[1]{E\left[ #1 \right]}
\providecommand{\cond}[2]{#1\middle|#2}
\providecommand{\fourier}{\overset{\mathcal{F}}{ \rightleftharpoons}}
\newenvironment{amatrix}[1]{%
  \left(\begin{array}{@{}*{#1}{c}|c@{}}
}{%
  \end{array}\right)
}
%\providecommand{\hilbert}{\overset{\mathcal{H}}{ \rightleftharpoons}}
%\providecommand{\system}{\overset{\mathcal{H}}{ \longleftrightarrow}}
	%\newcommand{\solution}[2]{\textbf{Solution:}{#1}}
\newcommand{\solution}{\noindent \textbf{Solution: }}
\newcommand{\cosec}{\,\text{cosec}\,}
\providecommand{\dec}[2]{\ensuremath{\overset{#1}{\underset{#2}{\gtrless}}}}
\newcommand{\myvec}[1]{\ensuremath{\begin{pmatrix}#1\end{pmatrix}}}
\newcommand{\mydet}[1]{\ensuremath{\begin{vmatrix}#1\end{vmatrix}}}
\newcommand{\myaugvec}[2]{\ensuremath{\begin{amatrix}{#1}#2\end{amatrix}}}
\providecommand{\rank}{\text{rank}}
\providecommand{\pr}[1]{\ensuremath{\Pr\left(#1\right)}}
\providecommand{\qfunc}[1]{\ensuremath{Q\left(#1\right)}}
	\newcommand*{\permcomb}[4][0mu]{{{}^{#3}\mkern#1#2_{#4}}}
\newcommand*{\perm}[1][-3mu]{\permcomb[#1]{P}}
\newcommand*{\comb}[1][-1mu]{\permcomb[#1]{C}}
\providecommand{\qfunc}[1]{\ensuremath{Q\left(#1\right)}}
\providecommand{\gauss}[2]{\mathcal{N}\ensuremath{\left(#1,#2\right)}}
\providecommand{\diff}[2]{\ensuremath{\frac{d{#1}}{d{#2}}}}
\providecommand{\myceil}[1]{\left \lceil #1 \right \rceil }
\newcommand\figref{Fig.~\ref}
\newcommand\tabref{Table~\ref}
\newcommand{\sinc}{\,\text{sinc}\,}
\newcommand{\rect}{\,\text{rect}\,}
%%
%	%\newcommand{\solution}[2]{\textbf{Solution:}{#1}}
%\newcommand{\solution}{\noindent \textbf{Solution: }}
%\newcommand{\cosec}{\,\text{cosec}\,}
%\numberwithin{equation}{section}
%\numberwithin{equation}{subsection}
%\numberwithin{problem}{section}
%\numberwithin{definition}{section}
%\makeatletter
%\@addtoreset{figure}{problem}
%\makeatother

%\let\StandardTheFigure\thefigure
\let\vec\mathbf


\bibliographystyle{IEEEtran}


\vspace{3cm}

\title{
%	\logo{
EE23010 NCERT Exemplar
%	}
}
\author{Vishal A - EE22BTECH11057}

	
	

%\title{
%	\logo{Matrix Analysis through Octave}{\begin{center}\includegraphics[scale=.24]{tlc}\end{center}}{}{HAMDSP}
%}


% paper title
% can use linebreaks \\ within to get better formatting as desired
%\title{Matrix Analysis through Octave}
%
%
% author names and IEEE memberships
% note positions of commas and nonbreaking spaces ( ~ ) LaTeX will not break
% a structure at a ~ so this keeps an author's name from being broken across
% two lines.
% use \thanks{} to gain access to the first footnote area
% a separate \thanks must be used for each paragraph as LaTeX2e's \thanks
% was not built to handle multiple paragraphs
%

%\author{<-this % stops a space
%\thanks{}}
%}
% note the % following the last \IEEEmembership and also \thanks - 
% these prevent an unwanted space from occurring between the last author name
% and the end of the author line. i.e., if you had this:
% 
% \author{....lastname \thanks{...} \thanks{...} }
%                     ^------------^------------^----Do not want these spaces!
%
% a space would be appended to the last name and could cause every name on that
% line to be shifted left slightly. This is one of those "LaTeX things". For
% instance, "\textbf{A} \textbf{B}" will typeset as "A B" not "AB". To get
% "AB" then you have to do: "\textbf{A}\textbf{B}"
% \thanks is no different in this regard, so shield the last } of each \thanks
% that ends a line with a % and do not let a space in before the next \thanks.
% Spaces after \IEEEmembership other than the last one are OK (and needed) as
% you are supposed to have spaces between the names. For what it is worth,
% this is a minor point as most people would not even notice if the said evil
% space somehow managed to creep in.



% The paper headers
%\markboth{Journal of \LaTeX\ Class Files,~Vol.~6, No.~1, January~2007}%
%{Shell \MakeLowercase{\textit{et al.}}: Bare Demo of IEEEtran.cls for Journals}
% The only time the second header will appear is for the odd numbered pages
% after the title page when using the twoside option.
% 
% *** Note that you probably will NOT want to include the author's ***
% *** name in the headers of peer review papers.                   ***
% You can use \ifCLASSOPTIONpeerreview for conditional compilation here if
% you desire.




% If you want to put a publisher's ID mark on the page you can do it like
% this:
%\IEEEpubid{0000--0000/00\$00.00~\copyright~2007 IEEE}
% Remember, if you use this you must call \IEEEpubidadjcol in the second
% column for its text to clear the IEEEpubid mark.



% make the title area
\maketitle
\textbf{Question 61.2023}\\
Using the Ordinary Least Squares (OLS) method, a researcher estimated the relationship between initial salary (S) of MBA graduates and their cumulative grade point average (CGPA) as
\begin{align*}
\hat{S}_i = \hat{\beta}_0 + \hat{\beta}_1 \text{CGPA}_i ,  i = 1,2, \dots , 100
\end{align*}
where $\hat{\beta}_0 = 4543$ and $\hat{\beta}_1 = 645.08$. The standard errors of $\hat{\beta}_0$ and $\hat{\beta}_1$ are 921.79 and 70.01, respectively.\\
The t-statistic for testing the null hypothesis $\beta_1 = 0$ is\\
\fi
\solution
\begin{definition}[t-statistic]
The t-statistic is the ratio of the difference between the estimated value of a parameter from its hypothesized value to its standard error.
\begin{align}
t_{\hat{\beta}_1} &= \frac{\hat{\beta}_1-\beta_1}{SE\brak{\hat{\beta}_1}}
\end{align}
where,
\begin{itemize}
\item $\hat{\beta}_1$ is the point estimate.
\item $\beta_1$ is the hypothesized value.
\item $SE\brak{\hat{\beta}_1}$ standard error of the estimator.
\end{itemize}
\end{definition}
\begin{definition}[Standard error]
It is a measure of how much the statistic is likely to vary from the true value of the parameter it is estimating.
\begin{align}
SE(\hat{\beta_1}) = \sqrt{\frac{s^2}{n - 2}}
\end{align}
where,
\begin{itemize}
\item $s^2$ is the variance
\item $n$ is the sample size
\end{itemize}
\end{definition}
Given that $\hat{\beta}_1 = 645.08$ and $SE\brak{\hat{\beta}_1}=70.01$, we get
\begin{align}
t_{\hat{\beta}_1} &= \frac{645.08-0}{70.01}\\
t_{\hat{\beta}_1} &= 9.21
\end{align}















\item Let \{0.13, 0.12, 0.78, 0.51\} be a realization of a random sample of size 4 from a population with cumulative distribution function $F(.)$. Consider testing
\begin{align}
H_0 : F = F_0 \quad \text{against} \quad H_1 : F \ne F_0
\end{align}
where,
\begin{align}
    F_0(x) &= 
    \begin{cases}
        0 &  x<0  \\
        x & 0\le x<1 \\
        1 & x\ge 1
    \end{cases}
\end{align}
Let $D$ denote the Kolmogorov-Smirnov test statistic. If $P (D > 0.669) = 0.01$ under $H_0$ and
\begin{align}
    \psi &= 
    \begin{cases}
        1 &  \text{if } H_0 \text{ is accepted at level } 0.01 \\
        0 &  \text{otherwise} 
    \end{cases} 
\end{align}
then based on the given data, the observed value of $D + \psi$ (rounded off to two decimal places) equals
\hfill(GATE ST 2023)
\\
\iffalse
\let\negmedspace\undefined
\let\negthickspace\undefined
\documentclass[article]{IEEEtran}
       \def\inputGnumericTable{}                                 %%
\usepackage{cite}
\usepackage{amsmath,amssymb,amsfonts,amsthm}
\usepackage{algorithmic}
\usepackage{graphicx}
\usepackage{textcomp}
\usepackage{xcolor}
\usepackage{txfonts}
\usepackage{listings}
\usepackage{enumitem}
\usepackage{mathtools}
\usepackage{gensymb}
\usepackage[breaklinks=true]{hyperref}
\usepackage{tkz-euclide} % loads  TikZ and tkz-base
\usepackage{listings}
\renewcommand{\theenumi}{\Alph{enumi}}
%
%\usepackage{setspace}
%\usepackage{gensymb}
%\doublespacing
%\singlespacing

%\usepackage{graphicx}
%\usepackage{amssymb}
%\usepackage{relsize}
%\usepackage[cmex10]{amsmath}
%\usepackage{amsthm}
%\interdisplaylinepenalty=2500
%\savesymbol{iint}
%\usepackage{txfonts}
%\restoresymbol{TXF}{iint}
%\usepackage{wasysym}
%\usepackage{amsthm}
%\usepackage{iithtlc}
%\usepackage{mathrsfs}
%\usepackage{txfonts}
%\usepackage{stfloats}
%\usepackage{bm}
%\usepackage{cite}
%\usepackage{cases}
%\usepackage{subfig}
%\usepackage{xtab}
%\usepackage{longtable}
%\usepackage{multirow}
%\usepackage{algorithm}
%\usepackage{algpseudocode}
%\usepackage{enumitem}
%\usepackage{mathtools}
%\usepackage{tikz}
%\usepackage{circuitikz}
%\usepackage{verbatim}
%\usepackage{tfrupee}
%\usepackage{stmaryrd}
%\usetkzobj{all}
    \usepackage{color}                                            %%
    \usepackage{array}                                            %%
    \usepackage{longtable}                                        %%
    \usepackage{calc}                                             %%
    \usepackage{multirow}                                         %%
    \usepackage{hhline}                                           %%
    \usepackage{ifthen}                                           %%
 %optionally (for landscape tables embedded in another document): %%
    \usepackage{lscape}     
%\usepackage{multicol}
%\usepackage{chngcntr}
%\usepackage{enumerate}

%\usepackage{wasysym}
%\documentclass[conference]{IEEEtran}
%\IEEEoverridecommandlockouts
% The preceding line is only needed to identify funding in the first footnote. If that is unneeded, please comment it out.

\newtheorem{theorem}{Theorem}[section]
\newtheorem{problem}{Problem}
\newtheorem{proposition}{Proposition}[section]
\newtheorem{lemma}{Lemma}[section]
\newtheorem{corollary}[theorem]{Corollary}
\newtheorem{example}{Example}[section]
\newtheorem{definition}[problem]{Definition}
%\newtheorem{thm}{Theorem}[section] 
%\newtheorem{defn}[thm]{Definition}
%\newtheorem{algorithm}{Algorithm}[section]
%\newtheorem{cor}{Corollary}
\newcommand{\BEQA}{\begin{eqnarray}}
\newcommand{\EEQA}{\end{eqnarray}}
\newcommand{\define}{\stackrel{\triangle}{=}}
\theoremstyle{remark}
\newtheorem{rem}{Remark}

\begin{document}
\providecommand{\pr}[1]{\ensuremath{\Pr\left(#1\right)}}
\providecommand{\prt}[2]{\ensuremath{p_{#1}^{\left(#2\right)} }}        % own macro for this question
\providecommand{\qfunc}[1]{\ensuremath{Q\left(#1\right)}}
\providecommand{\sbrak}[1]{\ensuremath{{}\left[#1\right]}}
\providecommand{\lsbrak}[1]{\ensuremath{{}\left[#1\right.}}
\providecommand{\rsbrak}[1]{\ensuremath{{}\left.#1\right]}}
\providecommand{\brak}[1]{\ensuremath{\left(#1\right)}}
\providecommand{\lbrak}[1]{\ensuremath{\left(#1\right.}}
\providecommand{\rbrak}[1]{\ensuremath{\left.#1\right)}}
\providecommand{\cbrak}[1]{\ensuremath{\left\{#1\right\}}}
\providecommand{\lcbrak}[1]{\ensuremath{\left\{#1\right.}}
\providecommand{\rcbrak}[1]{\ensuremath{\left.#1\right\}}}
\newcommand{\sgn}{\mathop{\mathrm{sgn}}}
\providecommand{\abs}[1]{\left\vert#1\right\vert}
\providecommand{\res}[1]{\Res\displaylimits_{#1}} 
\providecommand{\norm}[1]{\left\lVert#1\right\rVert}
%\providecommand{\norm}[1]{\lVert#1\rVert}
\providecommand{\mtx}[1]{\mathbf{#1}}
\providecommand{\mean}[1]{E\left[ #1 \right]}
\providecommand{\cond}[2]{#1\middle|#2}
\providecommand{\fourier}{\overset{\mathcal{F}}{ \rightleftharpoons}}
\newenvironment{amatrix}[1]{%
  \left(\begin{array}{@{}*{#1}{c}|c@{}}
}{%
  \end{array}\right)
}
%\providecommand{\hilbert}{\overset{\mathcal{H}}{ \rightleftharpoons}}
%\providecommand{\system}{\overset{\mathcal{H}}{ \longleftrightarrow}}
	%\newcommand{\solution}[2]{\textbf{Solution:}{#1}}
\newcommand{\solution}{\noindent \textbf{Solution: }}
\newcommand{\cosec}{\,\text{cosec}\,}
\providecommand{\dec}[2]{\ensuremath{\overset{#1}{\underset{#2}{\gtrless}}}}
\newcommand{\myvec}[1]{\ensuremath{\begin{pmatrix}#1\end{pmatrix}}}
\newcommand{\mydet}[1]{\ensuremath{\begin{vmatrix}#1\end{vmatrix}}}
\newcommand{\myaugvec}[2]{\ensuremath{\begin{amatrix}{#1}#2\end{amatrix}}}
\providecommand{\rank}{\text{rank}}
\providecommand{\pr}[1]{\ensuremath{\Pr\left(#1\right)}}
\providecommand{\qfunc}[1]{\ensuremath{Q\left(#1\right)}}
	\newcommand*{\permcomb}[4][0mu]{{{}^{#3}\mkern#1#2_{#4}}}
\newcommand*{\perm}[1][-3mu]{\permcomb[#1]{P}}
\newcommand*{\comb}[1][-1mu]{\permcomb[#1]{C}}
\providecommand{\qfunc}[1]{\ensuremath{Q\left(#1\right)}}
\providecommand{\gauss}[2]{\mathcal{N}\ensuremath{\left(#1,#2\right)}}
\providecommand{\diff}[2]{\ensuremath{\frac{d{#1}}{d{#2}}}}
\providecommand{\myceil}[1]{\left \lceil #1 \right \rceil }
\newcommand\figref{Fig.~\ref}
\newcommand\tabref{Table~\ref}
\newcommand{\sinc}{\,\text{sinc}\,}
\newcommand{\rect}{\,\text{rect}\,}
%%
%	%\newcommand{\solution}[2]{\textbf{Solution:}{#1}}
%\newcommand{\solution}{\noindent \textbf{Solution: }}
%\newcommand{\cosec}{\,\text{cosec}\,}
%\numberwithin{equation}{section}
%\numberwithin{equation}{subsection}
%\numberwithin{problem}{section}
%\numberwithin{definition}{section}
%\makeatletter
%\@addtoreset{figure}{problem}
%\makeatother

%\let\StandardTheFigure\thefigure
\let\vec\mathbf

\bibliographystyle{IEEEtran}
\title{
%	\logo{
Assignment
%	}
}
\author{ Karthikeya hanu prakash kanithi (EE22BTECH11026)}
\maketitle
\parindent0px
\vspace{3cm}
Question : Let \{0.13, 0.12, 0.78, 0.51\} be a realization of a random sample of size 4 from a population with cumulative distribution function $F(.)$. Consider testing
\begin{align}
H_0 : F = F_0 \quad \text{against} \quad H_1 : F \ne F_0
\end{align}
where,
\begin{align}
    F_0(x) &= 
    \begin{cases}
        0 &  x<0  \\
        x & 0\le x<1 \\
        1 & x\ge 1
    \end{cases}
\end{align}
Let $D$ denote the Kolmogorov-Smirnov test statistic. If $P (D > 0.669) = 0.01$ under $H_0$ and
\begin{align}
    \psi &= 
    \begin{cases}
        1 &  \text{if } H_0 \text{ is accepted at level } 0.01 \\
        0 &  \text{otherwise} 
    \end{cases} 
\end{align}
then based on the given data, the observed value of $D + \psi$ (rounded off to two decimal places) equals
\fi
\solution 
Its given that random sample is of size 4, So 
\begin{align}
	n=4 \label{eq:st/65/1}
\end{align}
The cdf of the random sample is given as 
\begin{align}
    F_X(x) &= 
    \begin{cases}
        0 &  x<0  \\
        x & 0\le x<1 \\
        1 & x\ge 1
    \end{cases} \label{eq:st/65/2}
\end{align}
The empirical distribution function(edf) $G_n$ for n independent and identically distributed (i.i.d.) ordered observations $X_i$ is defined as
\begin{align}
	G_n(x) = \frac{\text{no of (elements in the sample} \le x)}{n} = \frac{1}{n} \sum_{i=1}^{n} 1(X_i \le x)  \label{eq:st/65/3}
\end{align}
where $1(A)$ is the indicator of event A and in \eqref{eq:st/65/3} it is defined as,
\begin{align}
	1(X_i \le x)=
	\begin{cases}
	1 & X_i \le x\\
	0 & \text{otherwise}
	\end{cases}
\end{align}
From \eqref{eq:st/65/1}, \eqref{eq:st/65/2} and \eqref{eq:st/65/3}, the edf for the given data will be
\begin{align}
  	G_n(0.13) = \frac{1}{4} \sum_{i=1}^{n} 1(X_i \le 0.13) = \frac{1}{2}	\\
  	G_n(0.12) = \frac{1}{4} \sum_{i=1}^{n} 1(X_i \le 0.12) = \frac{1}{4}\\
  	G_n(0.78) = \frac{1}{4} \sum_{i=1}^{n} 1(X_i \le 0.78) = 1\\
  	G_n(0.51) = \frac{1}{4} \sum_{i=1}^{n} 1(X_i \le 0.51) = \frac{3}{4}
\end{align}
The Kolmogorov–Smirnov statistic for a given cdf $F_X(x)$ is
\begin{align}
  	D_n = \sup\abs{G_n(x) - F_X(x)} 
\end{align}
The difference between cdf and edf for the given data will be (i.e., $\forall x \in \{0.13, 0.12, 0.78, 0.51\} $)
\begin{align}
  	G_n(0.13) - F_X(0.13) = 0.37\\
  	G_n(0.12) - F_X(0.12) = 0.25\\
  	G_n(0.78) - F_X(0.78) = 0.22\\
  	G_n(0.51) - F_X(0.51) = 0.24
\end{align}
Then 
\begin{align}
  	D_n = \sup(0.37,0.25,0.22,0.24) = 0.37 \label{eq:st/65/4}
\end{align}
Given that,
\begin{align}
  	P (D > 0.669) = 0.01
\end{align}
Then
\begin{align}
H_0 =
\begin{cases}
\text{accepted at level } 0.01 & \text{if } D_n \le 0.669 \\
\text{rejected at level } 0.01 & \text{if } D_n > 0.669
\end{cases}\label{eq:st/65/5}
\end{align}
From \eqref{eq:st/65/4} and \eqref{eq:st/65/5}; We can say that $H_0$ is accepted at level 0.01 and 
\begin{align}
  	\psi = 1 
\end{align}
$\therefore$ the value will be 
\begin{align}
  	\psi + D_n = 1 + 0.37 = 1.37
\end{align}


\end{enumerate}
