
\renewcommand{\theequation}{\theenumi}
\renewcommand{\thefigure}{\theenumi}
\renewcommand{\thetable}{\theenumi}
\begin{enumerate}[label=\thesection.\arabic*.,ref=\thesection.\theenumi]
\numberwithin{equation}{enumi}
\numberwithin{figure}{enumi}
\numberwithin{table}{enumi}


\item Consider a communication scheme where the binary valued signal X satisfies $P\{X = +1\} = 0.75$ and $P\{X = -1\} = 0.25$. The received signal $Y = X+Z$, where Z is a Gaussian random variable with zero mean and variance $\sigma^2$. The received signal Y is fed to the threshold detector. The output of the threshold detector $\hat X$ is:\\
{\centering $
{\hat X}= 
\begin{cases} 
+1 & Y> \tau \\
-1 & Y \leqslant \tau 
\end{cases}
$\\}
To achieve minimum probability of error $P\{\hat X \neq X\}$, the threshols $ \tau$ should be

\begin{enumerate}
\begin{multicols}{2}
\setlength\itemsep{2em}

\item strictly positive
\item zero
\item strictly negative
\item strictly positive, zero or strictly negative depending on the nonzero value of $\sigma ^2$

\end{multicols}
\end{enumerate}
%

\item 
\begin{center}
    \centering\underline{\textbf{Common Data for the following two Questions :}}
    \end{center}
    Let $X$ be a random variable with probability density function $f \in \{f_0,f_1\},$ where     
$ 
f_0(x)=
\begin{cases}
2x & 0<x<1 \\
0 & \text{otherwise}
\end{cases}
$ \\

$ 
f_1(x)=
\begin{cases}
3x^2 &  0<x<1 \\
0 & \text{otherwise}
\end{cases}
$\\

For testing the null hypothesis $H_0:f \equiv f_0$ against the alternative hypothesis $H_1:f \equiv f_1$ at level of significance $\alpha = 0.19$, the power of the most powerful test is

\begin{enumerate}
\begin{multicols}{2}
\setlength\itemsep{2em}

\item $ 0.729$
\item $ 0.271$
\item $ 0.615$
\item $ 0.385$

\end{multicols}
\end{enumerate}


\item The variance of the random variable $X$ is

\begin{enumerate}
\begin{multicols}{2}
\setlength\itemsep{2em}

\item $ \frac{1}{12}$
\item $ \frac{1}{4}$
\item $ \frac{7}{12}$
\item $ \frac{5}{12}$

\end{multicols}
\end{enumerate}


\item The covariance between the random variables $X$ and $Y$ is

\begin{enumerate}
\begin{multicols}{2}
\setlength\itemsep{2em}

\item $ \frac{1}{3}$\\
\item $ \frac{1}{4}$\\
\item $ \frac{1}{6}$\\
\item $ \frac{1}{12}$\\

\end{multicols}
\end{enumerate}


\item A screening test is carried out to detect a certain disease. It is found that $12\%$ of the positive
reports and $15\%$ of the negative reports are incorrect. Assuming that the probability of a
person getting positive report is 0.01, the probability that a person tested gets an incorrect
report is \dots
\\
\solution
Let $X \in \{0,1\}$ represent the random variable, where 0 represents the case where a person gets a positive report while 1 represents the case where a person gets a negative report. From the question, 
\begin{align}
    \Pr{(X=0)} = 0.01
    \\\Pr{(X=0)} + \Pr{(X=1)} = 1
    \\\Pr{(X=1)} = 1 - 0.01 = 0.99
\end{align}
Let $Y \in \{0,1\}$ represent the random variable, where 0 represents a correct report whereas 1 represents an incorrect report.

\begin{align}
    \Pr{(Y=1 | X=0)} = 12\% = 0.12
    \\\Pr{(Y=1 | X=1)} = 15\% = 0.15
\end{align}
Then, from total probability theorem,
\begin{multline}
    \Pr{(Y=1)} = \Pr{(Y=1, X=0)} 
    \\+ \Pr{(Y=1, X=1)}
\end{multline}
Using Bayes theorem,
\begin{multline}
    \Pr{(Y=1)} = \Pr{(Y=1 | X=0)} \times \Pr{(X=0)}
    \\ + \Pr{(Y=1 | X=1)} \times \Pr{(X=1)}
\end{multline}
    
\begin{align}
    \Pr{(Y=1)} = & 0.12 \times 0.01 + 0.15 \times 0.99
    \\ = & 0.0012 + 0.1485
    \\ = & 0.1497    
\end{align}

\item A diagnostic test for a certain disease is 90\% accurate. That is, the probability of a person having (respectively, not having) the disease tested positive (respectively, negative) is 0.9. Fifty percent of the population has the disease. What is the probability that a randomly chosen person has the disease given that the person tested negative?
\\
\solution
Let X and Y be two Bernoulli random variables such that X,Y$\in$\cbrak{0,1} and as given fifty percent of the population has the disease, the probability mass function of X is 
\begin{align}
    p_{X}(n) = \pr{X = n} = 
\begin{cases}
0.5 &  n=1
\\
0.5 & n=0
\\
0 & otherwise
\end{cases}\label{xe2016-8:1}
\end{align}
where X denotes the health status of a person(X=1 if person is healthy and X=0 if person is diseased) and Y denotes the diagnostic test result (Y=1 if it is positive and Y=0 if it is negative).
\\Given the probabilities of,
\begin{align}
    \pr{Y=1|X=0}=0.9\label{xe2016-8:2}
    \\\pr{Y=0|X=1}=0.9 \label{xe2016-8:3}
\end{align}
we need to find $\pr{X=0|Y=0}$,
\begin{align}
    \pr{X=0|Y=0}&=\frac{\pr{X=0\cap Y=0}}{\pr{Y=0}}\\
    \pr{X=0|Y=0}&=\frac{\pr{Y=0|X=0}\pr{X=0}}{\pr{Y=0}}\label{xe2016-8:4}
\end{align}
\begin{multline}
    \pr{Y=0}=\pr{Y=0|X=1}\pr{X=1}\\+\pr{Y=0|X=0}\pr{X=0}\label{xe2016-8:5}
\end{multline}
Using \eqref{xe2016-8:1},\eqref{xe2016-8:2} and \eqref{xe2016-8:3} in \eqref{xe2016-8:5},
\begin{align}
    \nonumber\pr{Y=0}&=0.9(0.5)+(1-0.9)0.5\\
    \pr{Y=0}&= 0.5\label{xe2016-8:6}
\end{align}
Using \eqref{xe2016-8:1},\eqref{xe2016-8:2} and \eqref{xe2016-8:6} in \eqref{xe2016-8:4}
\begin{align}
    \pr{X=0|Y=0}&=\frac{(1-0.9)0.5}{0.5}\\
    \pr{X=0|Y=0}&= 0.1
\end{align}

%
\item The probability density function of a random variable X is
\begin{equation}
f(x)=
\begin{cases}
\frac{1}{\lambda}e^{\brak{-\frac{x}{\lambda}}}, & x>0\\
0, & x\leq 0
\end{cases}
\end{equation}
where $\lambda>0.$ For testing the hypothesis $H_{0}:\lambda=3$ against $H_{1}:\lambda=5$, a test is given as "Reject $H_0$ if $X\geq 4.5$".The probability of type 1 error and power of the test are respectively: 
\begin{enumerate}
\begin{multicols}{2}
\setlength\itemsep{1em}
\item 0.1353 and 0.4966\\
\item 0.1827 and 0.379\\
\item 0.2021 and 0.4493\\
\item 0.2231 and 0.4066
\end{multicols}
\end{enumerate}
%
  \solution
  %
\begin{definition}
\label{Type 1 error}A type 1 error occurs if the null hypothesis $H_{0}$ is rejected even if it is true.
\end{definition}
\begin{definition}
\label{Power of the test}The probability that the alternative hypothesis $H_{1}$ is true is defined to be Power of a given test. 
\end{definition}
Given,
\begin{equation}
f_{X}(x)=
\begin{cases}
\frac{1}{\lambda}e^{\brak{-\frac{x}{\lambda}}}, & x>0\\
0, & x\leq 0
\end{cases}
\end{equation}
Let cumulative distribution function be $F_{X}(x)$ for a given $\lambda$.
Hence,
\begin{equation}
    F_{X}(x)=\int_{-\infty}^{x}f_{X}(a) da
\end{equation}
From the probability density function,
\begin{align}
    \implies F_X(4.5)&=\int_{-\infty}^{x}f_{X}(a) da\\
    &=\int_{0}^{4.5}\frac{1}{\lambda}e^{\brak{-\frac{a}{\lambda}}}da\\
    &=1-e^{-\frac{4.5}{\lambda}}
\end{align}
We need the probability for $X\geq 4.5$,hence required probability is,
\begin{align}
1-F_{X}(4.5)=e^{-\frac{4.5}{\lambda}}\label{eq:2.0.6}
\end{align}
From \eqref{eq:2.0.6} we get probability that the given null hypothesis$(H_{0})$ is true is,
\begin{align}
e^{-\frac{4.5}{3}}=0.2231.
\end{align}
$\therefore$ The \textbf{probability of type 1 error is 0.2231}.
From \eqref{eq:2.0.6},we get the required probability that the given alternative hypothesis($H_{1}$) is true is,
\begin{align}
e^{-\frac{4.5}{5}}=0.4066
\end{align}
$\therefore$ The \textbf{power of the test is 0.4066}

%
\item Let $Y_{1},Y_{2},...,Y_{15}$ be a random sample of size 15 from the probability density function 
\begin{align}
\tag{Eq:1}
    f_{y}(y)=3(1-y)^{2} , 0<y<1
\end{align}
Use the central limit theorem to approximate $P\brak{\frac{1}{8}<\Bar{Y}<\frac{3}{8}}$
%
\solution
  
The \textbf{central limit theorem} states that whenever a random sample of size n is taken from any distribution with mean and variance, then the sample mean will be approximately normally distributed with mean and variance. The larger the value of the sample size, the better the approximation to the normal.
\begin{align}
\tag{1.1}
    Z_{n}=\frac{\bar{Y}-\mu}{\frac{\sigma}{\sqrt{n}}}
    \label{ma1996-25:eq:1}
\end{align}
From equation \ref{ma1996-25:eq:1}
\begin{align}
    \tag{1.2}
    \bar{Y}=Z_{n}\brak{\frac{\sigma}{\sqrt{n}}}+\mu
\end{align}
\begin{align}
\tag{1.3}
\pr{\frac{1}{8}<\Bar{Y}<\frac{3}{8}}
&=\pr{\frac{1}{8}<Z_{n}\brak{\frac{\sigma}{\sqrt{n}}}+\mu<\frac{3}{8}}\\
\tag{1.4}  
&=\pr{\frac{\frac{1}{8}-\mu}{\frac{\sigma}{\sqrt{n}}}<Z_{n}<\frac{\frac{3}{8}-\mu}{\frac{\sigma}{\sqrt{n}}}}
\label{ma1996-25:eq;123}
\end{align}
$\bar{Y}$:Mean of the randomly selected 15 variables
\begin{align}
\tag{1.5}
    \bar{Y}=\frac{Y_{1}+Y_{2}+..Y_{15}}{15}
\end{align}
Mean of probability density function is
\begin{align}
\tag{1.6}
\mu&=\int_{-\infty}^{\infty}yf(y)dy\\
\tag{1.7}
    &=\int_{0}^{1}y\times 3(1-y)^{2}dy\\
\tag{1.8}
    &=\frac{1}{4}
\end{align}
Variance of probability density function is
\begin{align}
\tag{1.9}
\sigma^{2}&=E[y^{2}]-(E[y])^{2}\\
\tag{1.10}
\label{ma1996-25:eq;2}
      &=\brak{\int_{0}^{1}y^{2}f(y)dy} - \brak{\frac{1}{4}}^{2}
\end{align}
\begin{align}
\tag{1.11}
    \int_{0}^{1}y^{2}f(y)dy &= \int_{0}^{1}y^{2}\times3(1-y)^{2}dy\\
\tag{1.12}
                            &=3\int_{0}^{1}(y-y^{2})^{2}dy\\
\tag{1.13}
\label{ma1996-25:eq;3}
                            &=\frac{1}{10}
\end{align}
Substituting equation \ref{ma1996-25:eq;3} in equation \ref{ma1996-25:eq;2}
\begin{align}
\tag{1.14}
 \sigma^{2}&=\frac{1}{10}-\frac{1}{16}\\
\tag{1.15}
           &=\frac{3}{80}
\end{align}
Using Q function in equation \ref{ma1996-25:eq;123} we have,
\begin{align}
\notag
\pr{\frac{1}{8}<\Bar{Y}<\frac{3}{8}}
&=\pr{\frac{\frac{1}{8}-\mu}{\frac{\sigma}{\sqrt{n}}}<Z_{n}<\frac{\frac{3}{8}-\mu}{\frac{\sigma}{\sqrt{n}}}}\\
\tag{1.16}
&=\pr{\frac{\frac{1}{8}-\mu(y)}{\frac{\sigma}{\sqrt{n}}}<Z_{n}<\frac{\frac{3}{8}-\mu(y)}{\frac{\sigma}{\sqrt{n}}}}\\
\tag{1.17}
&=Q\brak{\frac{\frac{-1}{8}}{\sqrt{\frac{3}{80}}}}-Q\brak{\frac{\frac{1}{8}}{\sqrt{\frac{3}{80}}}}\\
\tag{1.18}
&=1-2Q\brak{\frac{\frac{1}{8}}{\sqrt{\frac{3}{80}}}}\\
\tag{1.19}                             
&=1-2Q\brak{0.645}\\
\tag{1.20}                             
&=0.9938
\end{align}
  %
  \item Let X be a non-constant positive Random Variable such that $E(X) = 9$.\\
  Then which of the following statements is True?
  \begin{enumerate}
  \item  $E\brak{\frac{1}{X+1}} > 0.1$ and $\pr{X \ge 10} \le 0.9$
  \item   $E\brak{\frac{1}{X+1}} < 0.1$ and $\pr{X \ge 10} \le 0.9$
  \item   $E\brak{\frac{1}{X+1}} > 0.1$ and $\pr{X \ge 10} > 0.9$
  \item   $E\brak{\frac{1}{X+1}} < 0.1$ and $\pr{X \ge 10} > 0.9$
  \end{enumerate}
  %
  \solution
    
Given, for X $>$ 0 ,$E(X) = 9$, $E\brak{\frac{1}{X+1}}$ can be estimated by Jensens's Inequality. \\
\textbf{pre - requisites:}\\
In general, $\phi(X)$ is a convex function iff:
\begin{equation*}
    \frac{d^2 \phi}{dX^2} \ge 0
\end{equation*}
\textbf{Jensen's Inequality:}\\
In the context of probability theory, it is generally stated in the following form: if X is a random variable and $\phi$ is a convex function, then
\begin{equation*}
\tag{1} \label{st2021-1:jenson}
    \phi(E(X)) \le E(\phi(X))
\end{equation*}
\begin{align*}
    \text{So for } \phi(X) &= \frac{1}{X+1}, \\
                \frac{d\phi}{dX} &= - \frac{1}{(X+1)^{2}} \\
                \tag{2} \label{st2021-1:phi}
                \frac{d^2 \phi}{dX^2} &= \frac{2}{(X+1)^{3}} 
    \implies \frac{d^2 \phi}{dX^2} \ge 0,(\because X>0 )
\end{align*}
by eq \eqref{st2021-1:jenson} and \eqref{st2021-1:phi}
\begin{align*}
    E\brak{\frac{1}{X+1}} &\ge \frac{1}{E(X)+1} \\
    \implies E\brak{\frac{1}{X+1}} &\ge \frac{1}{9 + 1} \\
    \tag{3} \label{st2021-1:Part 1}
    \implies E\brak{\frac{1}{X+1}} &\ge 0.1
\end{align*}
$\pr{X \ge 10}$ can be estimated by Markov's Inequality.\\
\textbf{Markov's Inequality:}
If X is a non-negative random variable and a $>$ 0, then the probability that X is at least a is at most the expectation of X divided by a. \\
Mathematically,
\begin{equation*}
    \tag{4} \label{st2021-1:markov}
    \pr{X \ge a} \le \frac{E(X)}{a}
\end{equation*}
by \eqref{st2021-1:markov} for a = 10
\begin{align*}
    \pr{X \ge 10} &\le \frac{E(X)}{10} \\
    \implies \pr{X \ge 10} &\le \frac{9}{10} \\
    \tag{5} \label{st2021-1:Part 2}
   \therefore \pr{X \ge 10} &\le 0.9
\end{align*}
So, from \eqref{st2021-1:Part 1} and \eqref{st2021-1:Part 2} \\
\begin{center}
     \boxed{\textbf{Option 1 is the Correct Answer}}
\end{center}
%
\item Let $X_{1},X_{2},X_{3},\cdots$ be a sequence of i.i.d uniform \brak{0,1} random variables. Then the value of 
\begin{align} \lim_{n\to\infty} \pr{-\ln{\brak{1-X_{1}}}-\cdots-\ln{\brak{1-X_{n}}} > n} 
\end{align} is equal to 
%
%
\\
\solution
  \begin{align}
f_{X_{i}}\brak{x} ={}&\begin{cases}
1 & 0<x<1\\
0 & \text{otherwise}
\end{cases}
\end{align}
Let $Y_{1},Y_{2},\cdots,$ be another sequence of random variables where $Y_{i} = -\ln{\brak{1-X_{i}}},  i=1,2,3,\cdots$
\\ 
\begin{align}
f_{Y_{i}}\brak{x}={}& \frac{f_{X_{i}}\brak{x}}{\frac{dY_{i}}{dX_{i}}}\\
f_{Y_{i}}\brak{x}={}&\begin{cases}
e^{-x} & x>0\\
0 & \text{otherwise}
\end{cases}
\end{align}
From the above probability function, we have all $Y_{i}'$s to be exponential random variables.\\
\begin{align}
Y_{i}\sim \text{Exp}\brak{1}\\
\Rightarrow \mu = 1, \sigma^2 = 1
\end{align}
The required probability is 
\begin{align}
\lim_{n\to\infty}\pr{\sum_{i=1}^{n}Y_{i}>n}\\
=\lim_{n\to\infty}\pr{\overline{Y_{n}}>1}
\end{align}
Consider \begin{align}
Z=\lim_{n\to\infty}\sqrt{n}\brak{\frac{\overline{Y_{n}}-\mu}{\sigma}}
\end{align}
\\ Since $\overline{Y_{n}}>1$, we have $Z>0$.\\
By central limit theorem, we have Z to be a standard normal distribution.
\begin{align}
Z \sim {\mathcal {N}}\brak{0 ,1}\\
\lim_{n\to\infty}\pr{\overline{Y_{n}}>1}={}&\pr{Z>0}\\
={}&\frac{1}{2}
\end{align}
\begin{align}
\therefore \lim_{n\to\infty} \pr{-\ln{\brak{1-X_{1}}}-\cdots-\ln{\brak{1-X_{n}}} > n}=0.5
\end{align}



\end{enumerate}