\iffalse
\let\negmedspace\undefined
\let\negthickspace\undefined
\documentclass[journal,12pt,onecolumn]{IEEEtran}
\usepackage{cite}
\usepackage{amsmath,amssymb,amsfonts,amsthm}
\usepackage{algorithmic}
\usepackage{graphicx}
\usepackage{textcomp}
\usepackage{xcolor}
\usepackage{txfonts}
\usepackage{listings}
\usepackage{enumitem}
\usepackage{mathtools}
\usepackage{gensymb}
\usepackage[breaklinks=true]{hyperref}
\usepackage{tkz-euclide} % loads  TikZ and tkz-base
\usepackage{listings}



\newtheorem{theorem}{Theorem}[section]
\newtheorem{problem}{Problem}
\newtheorem{proposition}{Proposition}[section]
\newtheorem{lemma}{Lemma}[section]
\newtheorem{corollary}[theorem]{Corollary}
\newtheorem{example}{Example}[section]
\newtheorem{definition}[problem]{Definition}
%\newtheorem{thm}{Theorem}[section] 
%\newtheorem{defn}[thm]{Definition}
%\newtheorem{algorithm}{Algorithm}[section]
%\newtheorem{cor}{Corollary}
\newcommand{\BEQA}{\begin{eqnarray}}
\newcommand{\EEQA}{\end{eqnarray}}
\newcommand{\define}{\stackrel{\triangle}{=}}
\theoremstyle{remark}
\newtheorem{rem}{Remark}
%\bibliographystyle{ieeetr}
\begin{document}
%
\providecommand{\pr}[1]{\ensuremath{\Pr\left(#1\right)}}
\providecommand{\prt}[2]{\ensuremath{p_{#1}^{\left(#2\right)} }}        % own macro for this question
\providecommand{\qfunc}[1]{\ensuremath{Q\left(#1\right)}}
\providecommand{\sbrak}[1]{\ensuremath{{}\left[#1\right]}}
\providecommand{\lsbrak}[1]{\ensuremath{{}\left[#1\right.}}
\providecommand{\rsbrak}[1]{\ensuremath{{}\left.#1\right]}}
\providecommand{\brak}[1]{\ensuremath{\left(#1\right)}}
\providecommand{\lbrak}[1]{\ensuremath{\left(#1\right.}}
\providecommand{\rbrak}[1]{\ensuremath{\left.#1\right)}}
\providecommand{\cbrak}[1]{\ensuremath{\left\{#1\right\}}}
\providecommand{\lcbrak}[1]{\ensuremath{\left\{#1\right.}}
\providecommand{\rcbrak}[1]{\ensuremath{\left.#1\right\}}}
\newcommand{\sgn}{\mathop{\mathrm{sgn}}}
\providecommand{\abs}[1]{\left\vert#1\right\vert}
\providecommand{\res}[1]{\Res\displaylimits_{#1}} 
\providecommand{\norm}[1]{\left\lVert#1\right\rVert}
%\providecommand{\norm}[1]{\lVert#1\rVert}
\providecommand{\mtx}[1]{\mathbf{#1}}
\providecommand{\mean}[1]{E\left[ #1 \right]}
\providecommand{\cond}[2]{#1\middle|#2}
\providecommand{\fourier}{\overset{\mathcal{F}}{ \rightleftharpoons}}
\newenvironment{amatrix}[1]{%
  \left(\begin{array}{@{}*{#1}{c}|c@{}}
}{%
  \end{array}\right)
}
%\providecommand{\hilbert}{\overset{\mathcal{H}}{ \rightleftharpoons}}
%\providecommand{\system}{\overset{\mathcal{H}}{ \longleftrightarrow}}
	%\newcommand{\solution}[2]{\textbf{Solution:}{#1}}
\newcommand{\solution}{\noindent \textbf{Solution: }}
\newcommand{\cosec}{\,\text{cosec}\,}
\providecommand{\dec}[2]{\ensuremath{\overset{#1}{\underset{#2}{\gtrless}}}}
\newcommand{\myvec}[1]{\ensuremath{\begin{pmatrix}#1\end{pmatrix}}}
\newcommand{\mydet}[1]{\ensuremath{\begin{vmatrix}#1\end{vmatrix}}}
\newcommand{\myaugvec}[2]{\ensuremath{\begin{amatrix}{#1}#2\end{amatrix}}}
\providecommand{\rank}{\text{rank}}
\providecommand{\pr}[1]{\ensuremath{\Pr\left(#1\right)}}
\providecommand{\qfunc}[1]{\ensuremath{Q\left(#1\right)}}
	\newcommand*{\permcomb}[4][0mu]{{{}^{#3}\mkern#1#2_{#4}}}
\newcommand*{\perm}[1][-3mu]{\permcomb[#1]{P}}
\newcommand*{\comb}[1][-1mu]{\permcomb[#1]{C}}
\providecommand{\qfunc}[1]{\ensuremath{Q\left(#1\right)}}
\providecommand{\gauss}[2]{\mathcal{N}\ensuremath{\left(#1,#2\right)}}
\providecommand{\diff}[2]{\ensuremath{\frac{d{#1}}{d{#2}}}}
\providecommand{\myceil}[1]{\left \lceil #1 \right \rceil }
\newcommand\figref{Fig.~\ref}
\newcommand\tabref{Table~\ref}
\newcommand{\sinc}{\,\text{sinc}\,}
\newcommand{\rect}{\,\text{rect}\,}
%%
%	%\newcommand{\solution}[2]{\textbf{Solution:}{#1}}
%\newcommand{\solution}{\noindent \textbf{Solution: }}
%\newcommand{\cosec}{\,\text{cosec}\,}
%\numberwithin{equation}{section}
%\numberwithin{equation}{subsection}
%\numberwithin{problem}{section}
%\numberwithin{definition}{section}
%\makeatletter
%\@addtoreset{figure}{problem}
%\makeatother

%\let\StandardTheFigure\thefigure
\let\vec\mathbf

\bibliographystyle{IEEEtran}


\vspace{3cm}



\bigskip

\renewcommand{\thefigure}{\theenumi}
\renewcommand{\thetable}{\theenumi}
%\renewcommand{\theequation}{\theenumi}
Q: Let $\{-1, -\frac{1}{2}, 1, \frac{5}{2}, 3\}$ be a realization of a random sample of size $5$ from a population having $N\left(\frac{1}{2}, \sigma^2\right)$ distribution, where $\sigma > 0$ is an unknown parameter. Let $T$ be an unbiased estimator of $\sigma^2$ whose variance attains the Cramer-Rao lower bound. Then, based on the above data, the realized value of $T$ (rounded off to two decimal places) equals
\\ \solution
\fi
\begin{definition}
Unbiased Estimator is defined as
\begin{align}
\text{E}(\hat{\sigma^2}) = {\sigma^2}
\end{align}
where, \(E(\hat{\sigma^2})\) represents the expected value of the estimator \(\hat{\sigma^2}\) and \(\sigma^2\) represents the true parameter
\end{definition}
\begin{definition}
The Cramér-Rao bound can be defined as follows:
\begin{align}
\text{Var}(\sigma^2) \geq \frac{1}{I(\sigma^2)}
\end{align}
where $I(\sigma^2)$ represents fisher information for the parameter $\sigma^2$.
Mathematically,
\begin{equation*}
I(\sigma^2) = -E\left [\frac{\partial^2}{\partial(\sigma)^2}\log P_{X}(X|\sigma^2)\right]
\end{equation*}
where, $E[\cdot]$ represents the expected value and $P_{X}(X|\sigma^2)$ is the p.d.f of random variable $X$ given the parameter $\sigma^2$. 
\end{definition}
$P_X(X|\sigma^2)$ is given by:
\begin{align}
P_X(X|\sigma^2) &= \frac{1}{2\pi\sigma^2} \exp\left(-\frac{(X-\frac{1}{2})^2}{2\sigma^2}\right)\\
\log p_X(X|\sigma^2) &= \log\left(\frac{1}{2\pi\sigma^2} \exp\left(-\frac{(X-\frac{1}{2})^2}{2\sigma^2}\right)\right)\\
&= -\frac{1}{2}\log(2\pi\sigma^2) - \frac{(X-\frac{1}{2})^2}{2\sigma^2}\\
\frac{\partial^2}{\partial(\sigma^2)^2}\log P_X(X|\sigma^2) &= \frac{1}{2\pi\sigma^2} - \frac{3(X-\frac{1}{2})^2}{\sigma^4}\\
I(\sigma^2) &= \frac{3}{\sigma^4} E[X^2] - \frac{3}{\sigma^4} E[X] + \frac{3}{4\sigma^4} - \frac{1}{2\pi\sigma^2}\\
E[X^2] &= \sigma^2 + \left(\frac{1}{2}\right)^2\\
E[X] &= \frac{1}{2}\\
\implies I(\sigma^2) &= \left(3 - \frac{1}{2\pi}\right) \frac{1}{\sigma^2}
\end{align}
Hence, Cramér-Rao bound is given as $\frac{\sigma^2}{\left(3 - \frac{1}{2\pi}\right)}$
\begin{definition}
Variance of $T$ attains Cramer-Rao lower bound\\
\(\implies\) $T$ has attained minimum possible variance and $T$ is an efficient estimator
\end{definition}
\begin{table}[h!]
 \begin{center}
    \begin{tabular}{|c|c|c|c|c|c|}
    \hline
    $X_i$ & -1 & $-\frac{1}{2}$ & 1 & $\frac{5}{2}$ & 3\\
    \hline
    $({X_i} - \mu)^{2}$ & $\frac{9}{4}$ & 1 & $\frac{1}{4}$ & 4 & $\frac{25}{4}$\\
    \hline
    \end{tabular}
    \end{center}
    \caption{Table 1}
  \label{tab:GATE/2023/ST/63/} 
\end{table}
Therefore,
\begin{align}
T &= \frac{\sum (X_i - \mu)^{2}}{n}\\
n &= 5\\
\mu &= \frac{1}{2}
\end{align}
\begin{align}
\sum (X_i - \mu)^{2} &= 13.75
\end{align}
Hence,
\begin{align}
T &= 2.75
\end{align}
Since, $T$ is an unbiased estimator of $\sigma^2$,
\begin{align}
\text {Cramér-Rao bound} &= \frac{T}{\left(3 - \frac{1}{2\pi}\right)}\\
&= 0.968
\end{align}
