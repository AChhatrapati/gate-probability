\let\negmedspace\undefined
\let\negthickspace\undefined
\documentclass[journal,12pt,twocolumn]{IEEEtran}
\usepackage{cite}
\usepackage{amsmath,amssymb,amsfonts,amsthm}
\usepackage{algorithmic}
\usepackage{graphicx}
\usepackage{textcomp}
\usepackage{xcolor}
\usepackage{txfonts}
\usepackage{listings}
\usepackage{enumitem}
\usepackage{mathtools}
\usepackage{gensymb}
\usepackage[breaklinks=true]{hyperref}
\usepackage{tkz-euclide} % loads  TikZ and tkz-base
\usepackage{listings}
\usepackage{gvv}
%
%\usepackage{setspace}
%\usepackage{gensymb}
%\doublespacing
%\singlespacing

%\usepackage{graphicx}
%\usepackage{amssymb}
%\usepackage{relsize}
%\usepackage[cmex10]{amsmath}
%\usepackage{amsthm}
%\interdisplaylinepenalty=2500
%\savesymbol{iint}
%\usepackage{txfonts}
%\restoresymbol{TXF}{iint}
%\usepackage{wasysym}
%\usepackage{amsthm}
%\usepackage{iithtlc}
%\usepackage{mathrsfs}
%\usepackage{txfonts}
%\usepackage{stfloats}
%\usepackage{bm}
%\usepackage{cite}
%\usepackage{cases}
%\usepackage{subfig}
%\usepackage{xtab}
%\usepackage{longtable}
%\usepackage{multirow}
%\usepackage{algorithm}
%\usepackage{algpseudocode}
%\usepackage{enumitem}
%\usepackage{mathtools}
%\usepackage{tikz}
%\usepackage{circuitikz}
%\usepackage{verbatim}
%\usepackage{tfrupee}
%\usepackage{stmaryrd}
%\usetkzobj{all}
%    \usepackage{color}                                            %%
%    \usepackage{array}                                            %%
%    \usepackage{longtable}                                        %%
%    \usepackage{calc}                                             %%
%    \usepackage{multirow}                                         %%
%    \usepackage{hhline}                                           %%
%    \usepackage{ifthen}                                           %%
  %optionally (for landscape tables embedded in another document): %%
%    \usepackage{lscape}     
%\usepackage{multicol}
%\usepackage{chngcntr}
%\usepackage{enumerate}

%\usepackage{wasysym}
%\documentclass[conference]{IEEEtran}
%\IEEEoverridecommandlockouts
% The preceding line is only needed to identify funding in the first footnote. If that is unneeded, please comment it out.

\newtheorem{theorem}{Theorem}[section]
\newtheorem{problem}{Problem}
\newtheorem{proposition}{Proposition}[section]
\newtheorem{lemma}{Lemma}[section]
\newtheorem{corollary}[theorem]{Corollary}
\newtheorem{example}{Example}[section]
\newtheorem{definition}[problem]{Definition}
%\newtheorem{thm}{Theorem}[section] 
%\newtheorem{defn}[thm]{Definition}
%\newtheorem{algorithm}{Algorithm}[section]
%\newtheorem{cor}{Corollary}
\newcommand{\BEQA}{\begin{eqnarray}}
\newcommand{\EEQA}{\end{eqnarray}}
\newcommand{\define}{\stackrel{\triangle}{=}}
\theoremstyle{remark}
\newtheorem{rem}{Remark}

%\bibliographystyle{ieeetr}
\begin{document}
%

\bibliographystyle{IEEEtran}


\vspace{3cm}

\title{
%	\logo{
Solution of GATE-ST 2023 Q18
%	}
}
\author{ SUJAL GUPTA - EE22BTECH11052
}	
%\title{
%	\logo{Matrix Analysis through Octave}{\begin{center}\includegraphics[scale=.24]{tlc}\end{center}}{}{HAMDSP}
%}


% paper title
% can use linebreaks \\ within to get better formatting as desired
%\title{Matrix Analysis through Octave}
%
%
% author names and IEEE memberships
% note positions of commas and nonbreaking spaces ( ~ ) LaTeX will not break
% a structure at a ~ so this keeps an author's name from being broken across
% two lines.
% use \thanks{} to gain access to the first footnote area
% a separate \thanks must be used for each paragraph as LaTeX2e's \thanks
% was not built to handle multiple paragraphs
%

%\author{<-this % stops a space
%\thanks{}}
%}
% note the % following the last \IEEEmembership and also \thanks - 
% these prevent an unwanted space from occurring between the last author name
% and the end of the author line. i.e., if you had this:
% 
% \author{....lastname \thanks{...} \thanks{...} }
%                     ^------------^------------^----Do not want these spaces!
%
% a space would be appended to the last name and could cause every name on that
% line to be shifted left slightly. This is one of those "LaTeX things". For
% instance, "\textbf{A} \textbf{B}" will typeset as "A B" not "AB". To get
% "AB" then you have to do: "\textbf{A}\textbf{B}"
% \thanks is no different in this regard, so shield the last } of each \thanks
% that ends a line with a % and do not let a space in before the next \thanks.
% Spaces after \IEEEmembership other than the last one are OK (and needed) as
% you are supposed to have spaces between the names. For what it is worth,
% this is a minor point as most people would not even notice if the said evil
% space somehow managed to creep in.



% The paper headers
%\markboth{Journal of \LaTeX\ Class Files,~Vol.~6, No.~1, January~2007}%
%{Shell \MakeLowercase{\textit{et al.}}: Bare Demo of IEEEtran.cls for Journals}
% The only time the second header will appear is for the odd numbered pages
% after the title page when using the twoside option.
% 
% *** Note that you probably will NOT want to include the author's ***
% *** name in the headers of peer review papers.                   ***
% You can use \ifCLASSOPTIONpeerreview for conditional compilation here if
% you desire.




% If you want to put a publisher's ID mark on the page you can do it like
% this:
%\IEEEpubid{0000--0000/00\$00.00~\copyright~2007 IEEE}
% Remember, if you use this you must call \IEEEpubidadjcol in the second
% column for its text to clear the IEEEpubid mark.



% make the title area
\maketitle

\newpage

%\tableofcontents

\bigskip

\renewcommand{\thefigure}{\theenumi}
\renewcommand{\thetable}{\theenumi}

Suppose that $X$ has the probability density function
\begin{align}
f(x)&=
\begin{cases}
\frac{\lambda^{\alpha}}{\Gamma(\alpha)}x^{\alpha - 1} e^{-\lambda x} & \lambda > 0\\
0 & otherwise\\
\end{cases}
\end{align}
where $\alpha > 0$ and $\lambda > 0$. Which one of the following statements is NOT true?
\begin{enumerate}
\item $E(X)$ exists for all $\alpha > 0 $ and $ \lambda > 0$
\item Variance of $X$ exists for all $\alpha > 0$ and $\lambda > 0$
\item $E(\frac{1}{X})$ exists for all $\alpha > 0$ and $\lambda > 0$
\item $E(ln(1+X))$ exists for all $\alpha > 0$ and $\lambda > 0$
\end{enumerate}
\solution
\begin{enumerate}
\item
{
\begin{align}
E(X)&= \int_{-\infty}^{\infty} xp_X(x)dx\\
&= \int_{0}^{\infty} x\frac{\lambda^{\alpha}}{\Gamma(\alpha)}x^{\alpha - 1} e^{-\lambda x}\\
&= \frac{\lambda^{\alpha}}{\Gamma(\alpha)} \int_{0}^{\infty}x^{\alpha} e^{-\lambda x}\\
\end{align}
since we know that 
\begin{align}
\int_0^\infty x^{\alpha - 1} e^{-\lambda x} dx = \frac{\Gamma(\alpha)}{\lambda^{\alpha}} \qquad \textrm{for } \lambda > 0, \alpha>0
\end{align}
\begin{align}
E(X)&= \frac{\lambda^{\alpha}}{\Gamma(\alpha)}\frac{\Gamma(\alpha+1)}{\lambda^{\alpha+1}}
\end{align}
Using the relation
\begin{align}
\Gamma(x+1) = \Gamma(x) x
\end{align}
\begin{align}
E(X)=\frac{\alpha}{\lambda}
\end{align}
Thus $E(X)$ exists for all $\alpha > 0 $ and $ \lambda > 0$.
}
\item{
\begin{align}
{Var}(X) = {E}(X^2) - {E}(X)^2 
\end{align}
\begin{align}
{E}(X^2) &= \int_{0}^{\infty} x^2 \frac{{\lambda}^{\alpha}}{\Gamma({\alpha})} x^{{\alpha}-1} e^{-{\lambda}x} \, {d}x \\
&=\int_{0}^{\infty} \frac{{\lambda}^{\alpha}}{\Gamma({\alpha})} x^{({\alpha}+2)-1} e^{-{\lambda}x} \, {d}x \\
&=\int_{0}^{\infty} \frac{1}{{\lambda}^2} \frac{{\lambda}^{{\alpha}+2}}{\Gamma({\alpha})} x^{({\alpha}+2)-1} e^{-{\lambda}x} \, {d}x 
\end{align}
\begin{align}
{E}(X^2) = \int_{0}^{\infty} \frac{{\alpha}({\alpha}+1)}{{\lambda}^2}  \frac{{\lambda}^{{\alpha}+2}}{\Gamma({\alpha}+2)} x^{({\alpha}+2)-1} e^{-{\lambda}x}{d}x
\end{align}
using the density of the gamma distribution, we get
\begin{align}
{E}(X^2) &= \frac{{\alpha}({\alpha}+1)}{{\lambda}^2} 
\end{align}
\begin{align}
{Var}(X) &= \frac{{\alpha}^2+{\alpha}}{{\lambda}^2} - {\frac{{\alpha}}{{\lambda}}} ^2 \\
&= \frac{{\alpha}}{{\lambda}^2}
\end{align}
Thus, Variance of $X$ exists for all $\alpha > 0$ and $\lambda > 0$
}
\item {
\begin{align}
E\brak{\frac{1}{X}}&= \int_{0}^{\infty} \frac{1}{x}\frac{\lambda^{\alpha}}{\Gamma(\alpha)}x^{\alpha - 1} e^{-\lambda x}\\
&= \frac{\lambda^{\alpha}}{\Gamma(\alpha)} \int_{0}^{\infty}x^{\alpha-2} e^{-\lambda x}
\end{align}
For this, $\alpha >1$ is a must condition. Hence C is not a correct option.
Hence C is the answer.
}
\item
{
For $E(ln(1+X))$,
\begin{align}
E(ln(1+X))&=E(X)-\frac{E(X^2)}{2}+\frac{E(X^4)}{4}-..
\end{align}
We write the general expression for $E(X^n)$
\begin{align}
E(X^n)&=\frac{\brak{\alpha}\brak{\alpha+1}...\brak{\alpha+n-1}}{{\lambda}^n}
\end{align}
So by applying the ratio test to check the convergence of the sequence
\begin{align}
\lim_{n \to \infty}\biggr \rvert \frac{a_{n+1}}{a_n}\biggr \rvert = L\\
\biggr \rvert \frac{E(X^{n+2})}{E(X^n)}\biggr \rvert&=\frac{\frac{\brak{\alpha}\brak{\alpha+1}...\brak{\alpha+n+1}}{{\lambda}^{n+2}}}{\frac{\brak{\alpha}\brak{\alpha+1}...\brak{\alpha+n-1}}{{\lambda}^n}}\\
&=\frac{\brak{\alpha+n}\brak{\alpha+n+1}}{{\lambda}^2}
%E(X^n)&=\frac{\brak{\alpha}\brak{\alpha+1}...\brak{\alpha+n-1}}{{\lambda}^n}\\
\end{align}
\begin{align}
\lim_{n \to \infty}\biggr \rvert\frac{E(X^{n+2})}{E(X^n)}\biggr \rvert=\infty
\end{align}
Thus $E(ln(1+X))$ generates a divergent function and hence $E(ln(1+X))$ does not exist for all $\alpha > 0$ and $ \lambda > 0$.
}
\end{enumerate}
\end{document}
