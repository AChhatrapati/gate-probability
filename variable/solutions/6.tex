\begin{definition}[Heaviside step function]
    \label{var/6/Heaviside step function}
    Heaviside step function $u(x)$ is 
    \begin{align*}
    u(x)=                
        \begin{cases}
        0 & x<0 \\
        1 & x\geq 0
        \end{cases}
    \end{align*}
    Using the Heaviside step function $u(x)$, a function $F(t)$ can be obtained whose output is $f(t)$ for the interval $[a,b)$ and $0$ everywhere else
    \begin{align*}
        F(t)=f(t)[u(t-a) - u(t-b)] \tag{1} \label{var/6/1}
    \end{align*}
    \end{definition}
    \begin{definition}[Dirac delta function]
    \label{var/6/Dirac delta function}
    Dirac delta function is the derivative of the Heaviside step function $u(x)$
    \begin{align*}
        \delta(x) = \frac{du(x)}{dx} \tag{2} \label{var/6/2}
    \end{align*}
    An important property of the Dirac delta function is 
    \begin{align*}
        \int_{-\infty}^{\infty}f(x)\delta(x-x_0)dx = f(x_0) \tag{3} \label{var/6/3}
    \end{align*}
    \end{definition}
    Using Definition \ref{var/6/Heaviside step function}, we get CDF $F(x)$ in terms of Heaviside step function $u(x)$
    \begin{multline*}
    F(x)=\frac{1}{4}\cbrak{u(x)-u(x-1)}+\frac{1}{3}\cbrak{u(x-1)-u(x-2)}+\\
    \frac{1}{2}\cbrak{u(x-2)-u\brak{x-\frac{11}{3}}}+u\brak{x-\frac{11}{3}}
    \end{multline*}
    \begin{align*}
    \implies F(x)=\frac{u(x)}{4}+\frac{u(x-1)}{12}+\frac{u(x-2)}{6}+\frac{u\brak{x-\frac{11}{3}}}{2} \tag{4} \label{var/6/4}
    \end{align*}
    Differentiating \eqref{var/6/4} and using Definition \ref{var/6/Dirac delta function}, we obtain PDF $f(x)$ 
    \begin{align*}
    f(x)=\frac{\delta(x)}{4}+\frac{\delta(x-1)}{12}+\frac{\delta(x-2)}{6}+\frac{\delta\brak{x-\frac{11}{3}}}{2} \tag{5} \label{var/6/5}
    \end{align*}
     Using \eqref{var/6/5}, we can state that Random Variable $X$ is discrete and it takes values at the points where $f(x) \to \infty$ \\
    \begin{align*}
    \therefore X \in \cbrak{0, 1, 2, \frac{11}{3}} \tag{6} \label{var/6/6}
    \end{align*}
    To obtain the PMF$\brak{p_X(k)}$ we use the formula
    \begin{align*}
    \brak{p_X(k)} = 
    \lim_{x\to k}\int_{k}^{x}f(x)dx  
    \tag{7} \label{var/6/7}
    \end{align*}
    \begin{definition}[PMF of Random Variable $X$]
    \label{var/6/PMF}
     The PMF$\brak{p_X(k)}$ using \eqref{var/6/7} is :
    \begin{align*}
    \brak{p_X(k)}=
    \begin{cases}
    \frac{1}{4} & \text{if } k=0 \\
    \frac{1}{12} & \text{if } k=1 \\
    \frac{1}{6} & \text{if } k=2 \\
    \frac{1}{2} & \text{if } k=\frac{11}{3} \\
    0 & \text{otherwise}
    \end{cases}
    \end{align*}
    \end{definition}
    To obtain $E(x)$ we use the formula
    \begin{align*}
    E(X) = \sum  x\times \brak{p_X(k)} \label{var/6/8} \tag{8}
    \end{align*}
     Therefore, using PMF we get
    \begin{align*}
    E(X)& = \brak{0\times \frac{1}{4}} + \brak{1\times \frac{1}{12}} + \brak{2\times \frac{1}{6}} + \brak{\frac{11}{3}\times \frac{1}{2}} \\
    \implies E(X) &= 2.25
    \end{align*}